%%
%% Automatically generated file from DocOnce source
%% (https://github.com/hplgit/doconce/)
%%
%%


%-------------------- begin preamble ----------------------

\documentclass[%
oneside,                 % oneside: electronic viewing, twoside: printing
final,                   % draft: marks overfull hboxes, figures with paths
10pt]{article}

\listfiles               %  print all files needed to compile this document

\usepackage{relsize,makeidx,color,setspace,amsmath,amsfonts,amssymb}
\usepackage[table]{xcolor}
\usepackage{bm,ltablex,microtype}

\usepackage[pdftex]{graphicx}

\usepackage[T1]{fontenc}
%\usepackage[latin1]{inputenc}
\usepackage{ucs}
\usepackage[utf8x]{inputenc}

\usepackage{lmodern}         % Latin Modern fonts derived from Computer Modern

% Hyperlinks in PDF:
\definecolor{linkcolor}{rgb}{0,0,0.4}
\usepackage{hyperref}
\hypersetup{
    breaklinks=true,
    colorlinks=true,
    linkcolor=linkcolor,
    urlcolor=linkcolor,
    citecolor=black,
    filecolor=black,
    %filecolor=blue,
    pdfmenubar=true,
    pdftoolbar=true,
    bookmarksdepth=3   % Uncomment (and tweak) for PDF bookmarks with more levels than the TOC
    }
%\hyperbaseurl{}   % hyperlinks are relative to this root

\setcounter{tocdepth}{2}  % levels in table of contents

% prevent orhpans and widows
\clubpenalty = 10000
\widowpenalty = 10000

% --- end of standard preamble for documents ---


% insert custom LaTeX commands...

\raggedbottom
\makeindex
\usepackage[totoc]{idxlayout}   % for index in the toc
\usepackage[nottoc]{tocbibind}  % for references/bibliography in the toc

%-------------------- end preamble ----------------------

\begin{document}

% matching end for #ifdef PREAMBLE

\newcommand{\exercisesection}[1]{\subsection*{#1}}


% ------------------- main content ----------------------



% ----------------- title -------------------------

\thispagestyle{empty}

\begin{center}
{\LARGE\bf
\begin{spacing}{1.25}
PHY321: Classical Mechanics 1
\end{spacing}
}
\end{center}

% ----------------- author(s) -------------------------

\begin{center}
{\bf Homework 6, due Monday  March 1${}^{}$} \\ [0mm]
\end{center}

\begin{center}
% List of all institutions:
\end{center}
    
% ----------------- end author(s) -------------------------

% --- begin date ---
\begin{center}
Feb 22, 2021
\end{center}
% --- end date ---

\vspace{1cm}


\paragraph{Practicalities about  homeworks and projects.}
\begin{enumerate}
\item You can work in groups (optimal groups are often 2-3 people) or by yourself. If you work as a group you can hand in one answer only if you wish. \textbf{Remember to write your name(s)}!

\item Homeworks are available ten days before the deadline.

\item How do I(we)  hand in?  You can hand in the paper and pencil exercises as a scanned document. For this homework this applies to exercises 1-5. Alternatively, you can hand in everyhting (if you are ok with typing mathematical formulae using say Latex) as a jupyter notebook at D2L. The numerical exercise(s) (exercise 6 here) should always be handed in as a jupyter notebook by the deadline at D2L. 
\end{enumerate}

\noindent
\paragraph{Introduction to homework 6.}
This week's sets of classical pen and paper and computational
exercises are again a continuation of the topics from the previous homework sets. We keep
discussing conservation laws, conservative forces, energy, momentum and angular momentum. These conservation laws are central in Physics and understanding them properly lays the foundation for understanding and analyzing more complicated physics problems.
The relevant reading background is
\begin{enumerate}
\item chapters 3 and 4 of Taylor (there are many good examples there) and

\item chapters 10-14 of Malthe-Sørenssen.

\item For exercise 4 will also need sections 5.1-5.3 of Taylor.
\end{enumerate}

\noindent
The numerical homework is based on what you did in homework 5 and you are going to test whether energy is conserved or not.


\paragraph{Exercise 1 (40pt), Summary of what we have done till now.}
Compile a summary of the material covered in this class so far thate
covers the major topics (Newton’s Laws, conservative and
nonconservative forces, and conservation of energy, momentum, and
angular momentum).  This summary should not only contain a list of
equations but should also include important concepts, numerical
elements, and mathematical methods, and show the connections between
different concepts.  This assignment is meant to help you review to
materials in the course in preparation for the first midterm project.
Make this summary as long as you need to throughly review all of then
covered topics.




\paragraph{Exercise 2 (10 pt), Conservative forces.}
Which of the following force are conservative?
\begin{itemize}
\item 2a (2pt) $\bm{F}=k(x\bm{e}_1+2y\bm{e}_2+3z\bm{e}_3)$ where $k$ is a consttant.

\item 2b (2pt) $\bm{F}=y\bm{e}_1+x\bm{e}_2+0\bm{e}_3$.

\item 2c (2pt) $\bm{F}=k(-y\bm{e}_1+x\bm{e}_2+0\bm{e}_3)$ where $k$ is a constant.

\item 2d (4pt) For those which are conservative, find the corresponding potential energy $V$ and verify that direct differentiation that $\bm{F}=-\bm{\nabla} V$.
\end{itemize}

\noindent
\paragraph{Exercise 3 (10 pt), The Lennard-Jones potential.}
\href{{https://en.wikipedia.org/wiki/Lennard-Jones_potential}}{The Lennard-Jones potential} is often used to describe
the interaction between two atoms or ions or molecules. If you end up doing materals science and molecular dynamics calculations, it is very likely that you will encounter this potential model.
The expression for the potential energy is
of the molecule is:
\[
V(r) = V_0\left((\frac{a}{r})^{12}-(\frac{b}{r})^{6}\right),
\]
where $V_0$, $a$ and $b$ are constants and the potential depends only on the relative distance between two objects
$i$ and $j$, that is $r=\vert\bm{r}_i-\bm{r}_j\vert=\sqrt{(x_i-x_j)^2+(y_i-y_j)^2+(z_i-z_j)^2}$.

\begin{itemize}
\item 3a (3pt) Sketch/plot the potential (choose some values for the constants in doing so).

\item 3b (3pt) Find and classify the equilibrium points.

\item 3c (4pt) What is the force acting on one of the objects (an atom for example) from the other object? Is this a conservative force?
\end{itemize}

\noindent
\paragraph{Exercise 4 (10 pt), particle in a new potential.}
Relevant reading here is Taylor chapter 5 and the lecture notes on oscillations. In particular, you will find useful  sections 5.1, 5.2, and 5.4. They contain all material needed to solve this exercise.

Consider a particle of mass $m$ moving in a one-dimensional potential,
\[
V(x)=-\alpha\frac{x^2}{2}+\beta\frac{x^4}{4}.
\]

\begin{itemize}
\item 4a (3pt) Plot the potential and discuss eventual equilibrium points. Is this a conservative force?

\item 4b (3pt) Compute the second derivative of the potential and find its miminum position(s). Using the Taylor expansion of the potential around its minimum (see Taylor section 5.1) to define a spring constant $k$. Use the spring constant to find the natural (angular) frequency $\omega_0=\sqrt{k/m}$. We call the new spring constant for  an effective spring constant. 

\item 4c (4pt) We ignore the second term in the potential energy and keep only the term proportional to the effective spring constant, that is a force $F\propto kx$. Find the acceleration and set up the differential equation.  Find the general analytical solution for these harmonic oscillations.  You don't need to find the constants in the general solution.
\end{itemize}

\noindent
\paragraph{Exercise 5 (30pt), Testing Energy Conservation.}
This week we will reuse our code from homework 5 and test that energy is conserved numerically.

\begin{itemize}
\item 5a (10 pt) Show that the \textbf{curl} of the gravitational force is zero and discuss whether the gravitational force is  a conservative force or not. Hint: see Taylor section 4.4.

\item 5b (20pt) Using the code from homework 5 (see also the solution), compute the kinetic and potential energies as functions of time. You will need the velocties and positions to compute these quantities. Remember to define initial positions and velocities, as well as the  initial kinetic and potential energies. Use Euler's methods, the Euler-Cromer method or the Velocity-Verlet
\end{itemize}

\noindent
method from the code you developed in homework 5 and study if energy is conserved over time. For a circular motion, are the potential and kinetic energies conserved separately? Comment your results.




\paragraph{Classical Mechanics Extra Credit Assignment: Scientific Writing and attending Talks.}
The following gives you an opportunity to earn \textbf{five extra credit
points} on each of the remaining homeworks and \textbf{ten extra credit points}
on the midterms and finals.  This assignment also covers an aspect of
the scientific process that is not taught in most undergraduate
programs: scientific writing.  Writing scientific reports is how
scientist communicate their results to the rest of the field.  Knowing
how to assemble a well written scientific report will greatly benefit
you in you upper level classes, in graduate school, and in the work
place.

The full information on extra credits is found at \href{{https://github.com/mhjensen/Physics321/blob/master/doc/Homeworks/ExtraCredits/}}{\nolinkurl{https://github.com/mhjensen/Physics321/blob/master/doc/Homeworks/ExtraCredits/}}. There you will also find examples on how to write a scientific article. 
Below you can also find a description on how to gain extra credits by attending scientific talks.


This assignment allows you to gain extra credit points by practicing
your scientific writing.  For each of the remaining homeworks you can
submit the specified section of a scientific report (written about the
numerical aspect of the homework) for five extra credit points on the
assignment.  For the two midterms and the final, submitting a full
scientific report covering the numerical analysis problem will be
worth ten extra points.  For credit the grader must be able to tell
that you put effort into the assignment (i.e.~well written, well
formatted, etc.).  If you are unfamiliar with writing scientific
reports, \href{{https://github.com/mhjensen/Physics321/blob/master/doc/Homeworks/ExtraCredits/IntroductionScientificWriting.md}}{see the information here}

The following table explains what aspect of a scientific report is due
with which homework.  You can submit the assignment in any format you
like, in the same document as your homework, or in a different one.
Remember to cite any external references you use and include a
reference list.  There are no length requirements, but make sure what
you turn in is complete and through.  If you have any questions,
please contact Julie Butler at butler@frib.msu.edu.


\begin{quote}
\begin{tabular}{ccc}
\hline
\multicolumn{1}{c}{ HW/Project } & \multicolumn{1}{c}{ Due Date } & \multicolumn{1}{c}{ Extra Credit Assignment } \\
\hline
HW 3                       & 2-8                        & Abstract                   \\
HW 4                       & 2-15                       & Introduction               \\
HW 5                       & 2-22                       & Methods                    \\
HW 6                       & 3-1                        & Results and Discussion     \\
\textbf{Midterm 1}         & \textbf{3-12}              & \emph{Full Written Report} \\
HW 7                       & 3-22                       & Abstract                   \\
HW 8                       & 3-29                       & Introduction               \\
HW 9                       & 4-5                        & Results and Discussion     \\
\textbf{Midterm 2|} _4-16_ & \emph{Full Written Report} \\
HW 10                      & 4-26                       & Abstract                   \\
\textbf{Final}             & \textbf{4-30}              & \emph{Full Written Report} \\
\hline
\end{tabular}
\end{quote}

\noindent

You can also gain extra credits if you attend scientific talks.
This is described here.


\paragraph{Integrating Classwork With Research.}
This opportunity will allow you to earn up to 5 extra credit points on a Homework per week. These points can push you above 100\% or help make up for missed exercises.
In order to earn all points you must:

\begin{enumerate}
\item Attend an MSU research talk (recommended research oriented Clubs is  provided below)

\item Summarize the talk using at least 150 words

\item Turn in the summary along with your Homework.
\end{enumerate}

\noindent
Approved talks:
Talks given by researchers through the following clubs:
\begin{itemize}
\item Research and Idea Sharing Enterprise (RAISE)​: Meets Wednesday Nights Society for Physics Students (SPS)​: Meets Monday Nights

\item Astronomy Club​: Meets Monday Nights

\item Facility For Rare Isotope Beam (FRIB) Seminars: ​Occur multiple times a week
\end{itemize}

\noindent
If you have any questions please consult Jeremy Rebenstock, rebensto@msu.edu.

All the material on extra credits is at \href{{https://github.com/mhjensen/Physics321/blob/master/doc/Homeworks/ExtraCredits/}}{\nolinkurl{https://github.com/mhjensen/Physics321/blob/master/doc/Homeworks/ExtraCredits/}}. 






% ------------------- end of main content ---------------

\end{document}

