%%
%% Automatically generated file from DocOnce source
%% (https://github.com/doconce/doconce/)
%% doconce format latex hw6.do.txt --no_mako
%%
% #ifdef PTEX2TEX_EXPLANATION
%%
%% The file follows the ptex2tex extended LaTeX format, see
%% ptex2tex: https://code.google.com/p/ptex2tex/
%%
%% Run
%%      ptex2tex myfile
%% or
%%      doconce ptex2tex myfile
%%
%% to turn myfile.p.tex into an ordinary LaTeX file myfile.tex.
%% (The ptex2tex program: https://code.google.com/p/ptex2tex)
%% Many preprocess options can be added to ptex2tex or doconce ptex2tex
%%
%%      ptex2tex -DMINTED myfile
%%      doconce ptex2tex myfile envir=minted
%%
%% ptex2tex will typeset code environments according to a global or local
%% .ptex2tex.cfg configure file. doconce ptex2tex will typeset code
%% according to options on the command line (just type doconce ptex2tex to
%% see examples). If doconce ptex2tex has envir=minted, it enables the
%% minted style without needing -DMINTED.
% #endif

% #define PREAMBLE

% #ifdef PREAMBLE
%-------------------- begin preamble ----------------------

\documentclass[%
oneside,                 % oneside: electronic viewing, twoside: printing
final,                   % draft: marks overfull hboxes, figures with paths
10pt]{article}

\listfiles               %  print all files needed to compile this document

\usepackage{relsize,makeidx,color,setspace,amsmath,amsfonts,amssymb}
\usepackage[table]{xcolor}
\usepackage{bm,ltablex,microtype}

\usepackage[pdftex]{graphicx}

\usepackage[T1]{fontenc}
%\usepackage[latin1]{inputenc}
\usepackage{ucs}
\usepackage[utf8x]{inputenc}

\usepackage{lmodern}         % Latin Modern fonts derived from Computer Modern

% Hyperlinks in PDF:
\definecolor{linkcolor}{rgb}{0,0,0.4}
\usepackage{hyperref}
\hypersetup{
    breaklinks=true,
    colorlinks=true,
    linkcolor=linkcolor,
    urlcolor=linkcolor,
    citecolor=black,
    filecolor=black,
    %filecolor=blue,
    pdfmenubar=true,
    pdftoolbar=true,
    bookmarksdepth=3   % Uncomment (and tweak) for PDF bookmarks with more levels than the TOC
    }
%\hyperbaseurl{}   % hyperlinks are relative to this root

\setcounter{tocdepth}{2}  % levels in table of contents

% prevent orhpans and widows
\clubpenalty = 10000
\widowpenalty = 10000

% --- end of standard preamble for documents ---


% insert custom LaTeX commands...

\raggedbottom
\makeindex
\usepackage[totoc]{idxlayout}   % for index in the toc
\usepackage[nottoc]{tocbibind}  % for references/bibliography in the toc

%-------------------- end preamble ----------------------

\begin{document}

% matching end for #ifdef PREAMBLE
% #endif

\newcommand{\exercisesection}[1]{\subsection*{#1}}


% ------------------- main content ----------------------



% ----------------- title -------------------------

\thispagestyle{empty}

\begin{center}
{\LARGE\bf
\begin{spacing}{1.25}
PHY321: Classical Mechanics 1
\end{spacing}
}
\end{center}

% ----------------- author(s) -------------------------

\begin{center}
{\bf Homework 6, due Friday March 17${}^{}$} \\ [0mm]
\end{center}

\begin{center}
% List of all institutions:
\end{center}
    
% ----------------- end author(s) -------------------------

% --- begin date ---
\begin{center}
Mar 12, 2023
\end{center}
% --- end date ---

\vspace{1cm}


\paragraph{Practicalities about  homeworks and projects.}
\begin{enumerate}
\item You can work in groups (optimal groups are often 2-3 people) or by yourself. If you work as a group you can hand in one answer only if you wish. \textbf{Remember to write your name(s)}!

\item Homeworks are available ten days before the deadline.

\item How do I(we)  hand in?  You can hand in the paper and pencil exercises as a scanned document. For this homework this applies to exercises 1-5. Alternatively, you can hand in everyhting (if you are ok with typing mathematical formulae using say Latex) as a jupyter notebook at D2L. The numerical exercise(s) (exercise 6 here) should always be handed in as a jupyter notebook by the deadline at D2L. 
\end{enumerate}

\noindent
\paragraph{Introduction to homework 6.}
This week's sets of classical pen and paper and computational
exercises are again a continuation of the topics from the previous homework sets and the first midterm. We keep
discussing conservation laws, conservative forces, energy, momentum and angular momentum. These conservation laws are central in Physics and understanding them properly lays the foundation for understanding and analyzing more complicated physics problems.
The relevant reading background is
\begin{enumerate}
\item chapters 3 and 4 of Taylor (there are many good examples there) and

\item chapters 10-14 of Malthe-Sørenssen.

\item For exercise 4 will also need sections 5.1-5.6 of Taylor.
\end{enumerate}

\noindent
The numerical homework is based on what you did in homework 5 and/or the first midterm.

There are also two optional exercises, one is a simple survey after the midterm. We would love hearing back from you about how the course is progressing.
The other exercise is an optional exercises on harmonic oscillations.

\paragraph{Exercise 1 (40pt), Summary of what we have done till now (and getting started after spring break).}
Compile a summary of the material covered in this class so far that
covers the major topics (Newton’s Laws, conservative and
nonconservative forces, and conservation of energy, momentum, and
angular momentum and harmonic oscillations).  This summary should not only contain a list of
equations but should also include important concepts, numerical
elements, and mathematical methods, and show the connections between
different concepts.  
Make this summary as long as you need to throughly review all of the
covered topics.

\paragraph{Exercise 2 (10 pt), Conservative forces.}
Which of the following force are conservative?
\begin{itemize}
\item 2a (2pt) $\bm{F}=k(x\bm{e}_1+2y\bm{e}_2+3z\bm{e}_3)$ where $k$ is a consttant.

\item 2b (2pt) $\bm{F}=y\bm{e}_1+x\bm{e}_2+0\bm{e}_3$.

\item 2c (2pt) $\bm{F}=k(-y\bm{e}_1+x\bm{e}_2+0\bm{e}_3)$ where $k$ is a constant.

\item 2d (4pt) For those which are conservative, find the corresponding potential energy $V$ and verify that direct differentiation gives $\bm{F}=-\bm{\nabla}V$.
\end{itemize}

\noindent
\paragraph{Exercise 3 (10 pt), The Lennard-Jones potential.}
\href{{https://en.wikipedia.org/wiki/Lennard-Jones_potential}}{The Lennard-Jones potential} is often used to describe
the interaction between two atoms or ions or molecules. If you end up doing materals science and molecular dynamics calculations, it is very likely that you will encounter this potential model.
The expression for the potential energy is
of the molecule is:
\[
V(r) = V_0\left((\frac{a}{r})^{12}-(\frac{b}{r})^{6}\right),
\]
where $V_0$, $a$ and $b$ are constants and the potential depends only on the relative distance between two objects
$i$ and $j$, that is $r=\vert\bm{r}_i-\bm{r}_j\vert=\sqrt{(x_i-x_j)^2+(y_i-y_j)^2+(z_i-z_j)^2}$.

\begin{itemize}
\item 3a (3pt) Sketch/plot the potential (choose some values for the constants in doing so).

\item 3b (3pt) Find and classify the equilibrium point(s).

\item 3c (4pt) What is the force acting on one of the objects (an atom for example) from the other object? Is this a conservative force?
\end{itemize}

\noindent
\paragraph{Exercise 4 (50 pt+optional 30pts), particle/object in a harmonic oscillator  potential.}
In the first midterm we looked at an
object/particle moving in a potential which resulted in harmonic
oscillations.  The aim here is to summarize in more detail the
material from harmonic oscillations.

Relevant reading here is Taylor chapter 5 and the lecture notes on oscillations. 

We will consider a particle of mass $m$ moving in a one-dimensional potential,
\[
V(x)=k\frac{x^2}{2},
\]
where $k$ is a parameter.

We will limit ourselves to a one-dimensional system. You will need to select values for the initial conditions and the various parameters $k$, $m$, $b$, $\omega$ and $F_0$ discussed here.

\begin{itemize}
\item 4a (20pt)  Assume no driving force first and add a drag force $-bv$, where $v$ is the velocity. Find the forces acting on the object. Find the analytical solutions to the equations of motion. Discuss the three cases: \textbf{underdamping}, \textbf{critical damping} and \textbf{overdamping}.

\item 4b (5pt) Scale your equations of motion in terms of a dimensionless time $\tau = \omega_0 t$, where $t$ is time and $\omega_0^2=k/m$ is the so-called natural frequency. 

\item 4c (25pt) You can use your codes from either the first midterm or from homework 5.  Study numerically the three cases from 4a, that is the underdamped motion, the critically damped one and finally the overdamped one. Compare your numerical results with the analytical ones from 4a. Discuss your results. You can use the Euler-Cromer method, or the Velocity-Verlet method or the Runge-Kutta methods discussed during the lectures, see for example \href{{https://mhjensen.github.io/Physics321/doc/pub/week8/html/week8.html}}{\nolinkurl{https://mhjensen.github.io/Physics321/doc/pub/week8/html/week8.html}}. Alternatively, you could use the \textbf{odeint} solvers included functionality in Python, see \href{{https://docs.scipy.org/doc/scipy/reference/generated/scipy.integrate.odeint.html}}{\nolinkurl{https://docs.scipy.org/doc/scipy/reference/generated/scipy.integrate.odeint.html}}. Give a short argument about the numerical algorithm you ended up using.  

\item 4d \textbf{Optional exercise} (30pt additional score) We add now a driving force $F=F_0\cos{(\omega t}$. Find the particular solution and its analytical solution. Include this force in your code (remember to scale the equations) and compare your numerical results with the analytical results. Discuss your results. How does the system evolve over time with a given frequency $\omega$ for the driving force?   Is energy conserved? If not, why? 
\end{itemize}

\noindent
\paragraph{Additional Bonus Exercise (10pt).}
You don't need to do this exercise, but it gives you a bonus score of 10 points.

This time the additional bonus exercise is a simple survey. We are now moving
into our second half of the semester and we would very much have your feedback on how
things are functioning so that we can improve and correct. 

\begin{itemize}
\item Is the weekly load with paper and pencil exercises and the numerical exercises reasonable?

\item Is there enough material (lectures and lecture material) to get you started with the exercises?  We are thinking of both the paper and pencil and the numerical exercises?

\item Is the pace during the lectures reasonable? And do the lectures link well with the exercises?

\item In the beginning there will always be some elements of repetition of material many of you have seen before. Do you find the choice of material the first 4 weeks adequate? Too easy? Too difficult?

\item For those of you who have taken CMSE 201 Introduction to Computational Modeling, do you feel the material taught there links well with the exercises you have done in this course? Is there is something we are missing?

\item Any other topic you would like to comment on?
\end{itemize}

\noindent
\paragraph{Classical Mechanics Extra Credit Assignment: Scientific Writing and attending Talks.}
The following gives you an opportunity to earn \textbf{five extra credit
points} on each of the remaining homeworks and \textbf{ten extra credit points}
on the midterms and finals.  This assignment also covers an aspect of
the scientific process that is not taught in most undergraduate
programs: scientific writing.  Writing scientific reports is how
scientist communicate their results to the rest of the field.  Knowing
how to assemble a well written scientific report will greatly benefit
you in you upper level classes, in graduate school, and in the work
place.

The full information on extra credits is found at \href{{https://github.com/mhjensen/Physics321/blob/master/doc/Homeworks/ExtraCredits/}}{\nolinkurl{https://github.com/mhjensen/Physics321/blob/master/doc/Homeworks/ExtraCredits/}}. There you will also find examples on how to write a scientific article. 
Below you can also find a description on how to gain extra credits by attending scientific talks.

This assignment allows you to gain extra credit points by practicing
your scientific writing.  For each of the remaining homeworks you can
submit the specified section of a scientific report (written about the
numerical aspect of the homework) for five extra credit points on the
assignment.  For the two midterms and the final, submitting a full
scientific report covering the numerical analysis problem will be
worth ten extra points.  For credit the grader must be able to tell
that you put effort into the assignment (i.e.~well written, well
formatted, etc.).  If you are unfamiliar with writing scientific
reports, \href{{https://github.com/mhjensen/Physics321/blob/master/doc/Homeworks/ExtraCredits/IntroductionScientificWriting.md}}{see the information here}

The following table explains what aspect of a scientific report is due
with which homework.  You can submit the assignment in any format you
like, in the same document as your homework, or in a different one.
Remember to cite any external references you use and include a
reference list.  There are no length requirements, but make sure what
you turn in is complete and through.  If you have any questions,
please contact Julie Butler at butler@frib.msu.edu.


\begin{quote}
\begin{tabular}{ccc}
\hline
\multicolumn{1}{c}{ HW/Project } & \multicolumn{1}{c}{ Due Date } & \multicolumn{1}{c}{ Extra Credit Assignment } \\
\hline
HW 3                       & 2-8                        & Abstract                   \\
HW 4                       & 2-15                       & Introduction               \\
HW 5                       & 2-22                       & Methods                    \\
HW 6                       & 3-1                        & Results and Discussion     \\
\textbf{Midterm 1}         & \textbf{3-12}              & \emph{Full Written Report} \\
HW 7                       & 3-22                       & Abstract                   \\
HW 8                       & 3-29                       & Introduction               \\
HW 9                       & 4-5                        & Results and Discussion     \\
\textbf{Midterm 2|} _4-16_ & \emph{Full Written Report} \\
HW 10                      & 4-26                       & Abstract                   \\
\textbf{Final}             & \textbf{4-30}              & \emph{Full Written Report} \\
\hline
\end{tabular}
\end{quote}

\noindent
You can also gain extra credits if you attend scientific talks.
This is described here.

\paragraph{Integrating Classwork With Research.}
This opportunity will allow you to earn up to 5 extra credit points on a Homework per week. These points can push you above 100\% or help make up for missed exercises.
In order to earn all points you must:

\begin{enumerate}
\item Attend an MSU research talk (recommended research oriented Clubs is  provided below)

\item Summarize the talk using at least 150 words

\item Turn in the summary along with your Homework.
\end{enumerate}

\noindent
Approved talks:
Talks given by researchers through the following clubs:
\begin{itemize}
\item Research and Idea Sharing Enterprise (RAISE)​: Meets Wednesday Nights Society for Physics Students (SPS)​: Meets Monday Nights

\item Astronomy Club​: Meets Monday Nights

\item Facility For Rare Isotope Beam (FRIB) Seminars: ​Occur multiple times a week
\end{itemize}

\noindent
All the material on extra credits is at \href{{https://github.com/mhjensen/Physics321/blob/master/doc/Homeworks/ExtraCredits/}}{\nolinkurl{https://github.com/mhjensen/Physics321/blob/master/doc/Homeworks/ExtraCredits/}}. 


% ------------------- end of main content ---------------

% #ifdef PREAMBLE
\end{document}
% #endif

