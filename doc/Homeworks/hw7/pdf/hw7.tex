%%
%% Automatically generated file from DocOnce source
%% (https://github.com/hplgit/doconce/)
%%
%%


%-------------------- begin preamble ----------------------

\documentclass[%
oneside,                 % oneside: electronic viewing, twoside: printing
final,                   % draft: marks overfull hboxes, figures with paths
10pt]{article}

\listfiles               %  print all files needed to compile this document

\usepackage{relsize,makeidx,color,setspace,amsmath,amsfonts,amssymb}
\usepackage[table]{xcolor}
\usepackage{bm,ltablex,microtype}

\usepackage[pdftex]{graphicx}

\usepackage[T1]{fontenc}
%\usepackage[latin1]{inputenc}
\usepackage{ucs}
\usepackage[utf8x]{inputenc}

\usepackage{lmodern}         % Latin Modern fonts derived from Computer Modern

% Hyperlinks in PDF:
\definecolor{linkcolor}{rgb}{0,0,0.4}
\usepackage{hyperref}
\hypersetup{
    breaklinks=true,
    colorlinks=true,
    linkcolor=linkcolor,
    urlcolor=linkcolor,
    citecolor=black,
    filecolor=black,
    %filecolor=blue,
    pdfmenubar=true,
    pdftoolbar=true,
    bookmarksdepth=3   % Uncomment (and tweak) for PDF bookmarks with more levels than the TOC
    }
%\hyperbaseurl{}   % hyperlinks are relative to this root

\setcounter{tocdepth}{2}  % levels in table of contents

% prevent orhpans and widows
\clubpenalty = 10000
\widowpenalty = 10000

% --- end of standard preamble for documents ---


% insert custom LaTeX commands...

\raggedbottom
\makeindex
\usepackage[totoc]{idxlayout}   % for index in the toc
\usepackage[nottoc]{tocbibind}  % for references/bibliography in the toc

%-------------------- end preamble ----------------------

\begin{document}

% matching end for #ifdef PREAMBLE

\newcommand{\exercisesection}[1]{\subsection*{#1}}


% ------------------- main content ----------------------



% ----------------- title -------------------------

\thispagestyle{empty}

\begin{center}
{\LARGE\bf
\begin{spacing}{1.25}
PHY321: Classical Mechanics 1
\end{spacing}
}
\end{center}

% ----------------- author(s) -------------------------

\begin{center}
{\bf Homework 7, due Monday  March 22${}^{}$} \\ [0mm]
\end{center}

\begin{center}
% List of all institutions:
\end{center}
    
% ----------------- end author(s) -------------------------

% --- begin date ---
\begin{center}
Mar 14, 2021
\end{center}
% --- end date ---

\vspace{1cm}


\paragraph{Practicalities about  homeworks and projects.}
\begin{enumerate}
\item You can work in groups (optimal groups are often 2-3 people) or by yourself. If you work as a group you can hand in one answer only if you wish. \textbf{Remember to write your name(s)}!

\item Homeworks are available  the week before the deadline. 

\item How do I(we)  hand in?  Due to the corona virus and many of you not being on campus, we recommend that you scan your handwritten notes and upload them to D2L. If you are ok with typing mathematical formulae using say Latex, you can hand in everything as a single jupyter notebook at D2L. The numerical exercise(s) should always be handed in as a jupyter notebook by the deadline at D2L. 
\end{enumerate}

\noindent
\paragraph{Introduction to homework 7.}
In this week's homework we will apply our insights about harmonic
oscillations and link this with our activity from the lecture on
Friday March 12. The relevant material to survey is chapter 5 of Taylor.

We have also added an exercise (exercise 2) related to our discussion of two-body problems. 
The relevant reading background for exercise 2 is sections 8.1-8.2 of Taylor.



\paragraph{Exercise 1 (80 pt), particle/object in a harmonic oscillator  potential.}
In the midterm and in exercise 4 of homework 6, we looked at an
object/particle moving in a potential which resulted in harmonic
oscillations.  The aim here is to summarize in more detail the
material from harmonic oscillations. See also the bonus exercise below
here (from the discussions of Friday March 12).


Relevant reading here is Taylor chapter 5 and the lecture notes on oscillations. 

We will consider a particle of mass $m$ moving in a one-dimensional potential,
\[
V(x)=k\frac{x^2}{2},
\]
where $k$ is a parameter.

We will limit ourselves to a one-dimensional system. You will need to select values for the initial conditions and the various parameters $k$, $m$, $b$, $\omega$ and $F_0$ discussed here.

\begin{itemize}
\item 1a (20pt)  Assume no driving force first and add a drag force $-bv$, where $v$ is the velocity. Find the forces acting on the object. Find the analytical solutions to the equations of motion. Discuss the three cases: \textbf{underdamping}, \textbf{critical damping} and \textbf{overdamping}.

\item 1b (5pt) Scale your equations of motion in terms of a dimensionless time $\tau = \omega_0 t$, where $t$ is time and $\omega_0^2=k/m$ is the so-called natural frequency. 

\item 1c (25pt) You can use your codes from either the first midterm or from homeworks 5 and/or 6.  Study numerically the three cases from 1a, that is the underdamped motion, the critically damped one and finally the overdamped one. Compare your numerical results with the analytical ones from 1a. Discuss your results. You can use the Euler-Cromer method, or the Velocity-Verlet method or the Runge-Kutta methods discussed during the lectures, see for example \href{{https://mhjensen.github.io/Physics321/doc/pub/week10/html/week10.html}}{\nolinkurl{https://mhjensen.github.io/Physics321/doc/pub/week10/html/week10.html}}. Alternatively, you could use the \textbf{odeint} solvers included functionality in Python, see \href{{https://docs.scipy.org/doc/scipy/reference/generated/scipy.integrate.odeint.html}}{\nolinkurl{https://docs.scipy.org/doc/scipy/reference/generated/scipy.integrate.odeint.html}}. Give a short argument about the numerical algorithm you ended up using.  

\item 1d (30pt) We add now a driving force $F=F_0\cos{(\omega t}$. Find the particular solution and its analytical solution. Include this force in your code (remember to scale the equations) and compare your numerical results with the analytical results. Discuss your results. How does the system evolve over time with a given frequency $\omega$ for the driving force?   Is energy conserved? If not, why? 
\end{itemize}

\noindent
\paragraph{Exercise 2 (20pt), Center-of-Mass and Relative Coordinates and Reference Frames.}
We define the two-body center-of-mass coordinate and relative coordinate by expressing the trajectories for
$\bm{r}_1$ and $\bm{r}_2$ into the center-of-mass coordinate
$\bm{R}_{\rm cm}$ 
\[
\bm{R}_{\rm cm}\equiv\frac{m_1\bm{r}_1+m_2\bm{r}_2}{m_1+m_2},
\]
and the relative coordinate 
\[
\bm{r}\equiv\bm{r}_1-\bm{r_2}.
\]
Here, we assume the two particles interact only with one another, so $\bm{F}_{12}=-\bm{F}_{21}$ (where $\bm{F}_{ij}$ is the force on $i$ due to $j$.

\begin{itemize}
\item 2a (5pt) Show that the equations of motion then become $\ddot{\bm{R}}_{\rm cm}=0$ and $\mu\ddot{\bm{r}}=\bm{F}_{12}$, with the reduced mass $\mu=m_1m_2/(m_1+m_2)$.
\end{itemize}

\noindent
The first expression simply states that the center of mass coordinate $\bm{R}_{\rm cm}$ moves at a fixed velocity. The second expression can be rewritten in terms of the reduced mass $\mu$.

\begin{itemize}
\item 2b (5pt) Show that the linear momenta for the center-of-mass $\bm{P}$ motion and the relative motion $\bm{q}$ are given by $\bm{P}=M\dot{\bm{R}}_{\rm cm}$ with $M=m_1+m_2$ and $\bm{q}=\mu\dot{\bm{r}}$.  The linear momentum of the relative motion is defined $\bm{q} = (m_2\bm{p}_1-m_1\bm{p}_2)/(m_1+m_2)$.

\item 2c (5pt) Show then the kinetic energy for two objects can then be written as
\end{itemize}

\noindent
\[
K=\frac{P^2}{2M}+\frac{q^2}{2\mu}.
\]

\begin{itemize}
\item 2d (5pt) Show that the total angular momentum for two-particles in the center-of-mass frame $\bm{R}=0$, is given by
\end{itemize}

\noindent
\[
\bm{L}=\bm{r}\times \mu\dot{\bm{r}}.
\]


\paragraph{Bonus exercise from lecture Friday March 12.}
This bonus exercise gives you an additional score of \textbf{10pt} and summarizes what we discussed during the lecture of March 12. It is useful in connection with solving exercise 1 as well.
\textbf{Your task is to write a summary page on the harmonic oscillator including all relevant equations, concepts, and especially how things are connected}. Your summary should be concise, not more than one page.


\paragraph{Classical Mechanics Extra Credit Assignment: Scientific Writing and attending Talks.}
The following gives you an opportunity to earn \textbf{five extra credit
points} on each of the remaining homeworks and \textbf{ten extra credit points}
on the midterms and finals.  This assignment also covers an aspect of
the scientific process that is not taught in most undergraduate
programs: scientific writing.  Writing scientific reports is how
scientist communicate their results to the rest of the field.  Knowing
how to assemble a well written scientific report will greatly benefit
you in you upper level classes, in graduate school, and in the work
place.

The full information on extra credits is found at \href{{https://github.com/mhjensen/Physics321/blob/master/doc/Homeworks/ExtraCredits/}}{\nolinkurl{https://github.com/mhjensen/Physics321/blob/master/doc/Homeworks/ExtraCredits/}}. There you will also find examples on how to write a scientific article. 
Below you can also find a description on how to gain extra credits by attending scientific talks.


This assignment allows you to gain extra credit points by practicing
your scientific writing.  For each of the remaining homeworks you can
submit the specified section of a scientific report (written about the
numerical aspect of the homework) for five extra credit points on the
assignment.  For the two midterms and the final, submitting a full
scientific report covering the numerical analysis problem will be
worth ten extra points.  For credit the grader must be able to tell
that you put effort into the assignment (i.e.~well written, well
formatted, etc.).  If you are unfamiliar with writing scientific
reports, \href{{https://github.com/mhjensen/Physics321/blob/master/doc/Homeworks/ExtraCredits/IntroductionScientificWriting.md}}{see the information here}

The following table explains what aspect of a scientific report is due
with which homework.  You can submit the assignment in any format you
like, in the same document as your homework, or in a different one.
Remember to cite any external references you use and include a
reference list.  There are no length requirements, but make sure what
you turn in is complete and through.  If you have any questions,
please contact Julie Butler at butler@frib.msu.edu.


\begin{quote}
\begin{tabular}{ccc}
\hline
\multicolumn{1}{c}{ HW/Project } & \multicolumn{1}{c}{ Due Date } & \multicolumn{1}{c}{ Extra Credit Assignment } \\
\hline
HW 3               & 2-8           & Abstract                   \\
HW 4               & 2-15          & Introduction               \\
HW 5               & 2-22          & Methods                    \\
HW 6               & 3-1           & Results and Discussion     \\
\textbf{Midterm 1} & \textbf{3-12} & \emph{Full Written Report} \\
HW 7               & 3-22          & Abstract                   \\
HW 8               & 3-29          & Introduction               \\
HW 9               & 4-5           & Results and Discussion     \\
\textbf{Midterm 2} & \textbf{4-16} & \emph{Full Written Report} \\
HW 10              & 4-26          & Abstract                   \\
\textbf{Final}     & \textbf{4-30} & \emph{Full Written Report} \\
\hline
\end{tabular}
\end{quote}

\noindent

You can also gain extra credits if you attend scientific talks.
This is described here.


\paragraph{Integrating Classwork With Research.}
This opportunity will allow you to earn up to 5 extra credit points on a Homework per week. These points can push you above 100\% or help make up for missed exercises.
In order to earn all points you must:

\begin{enumerate}
\item Attend an MSU research talk (recommended research oriented Clubs is  provided below)

\item Summarize the talk using at least 150 words

\item Turn in the summary along with your Homework.
\end{enumerate}

\noindent
Approved talks:
Talks given by researchers through the following clubs:
\begin{itemize}
\item Research and Idea Sharing Enterprise (RAISE)​: Meets Wednesday Nights Society for Physics Students (SPS)​: Meets Monday Nights

\item Astronomy Club​: Meets Monday Nights

\item Facility For Rare Isotope Beam (FRIB) Seminars: ​Occur multiple times a week
\end{itemize}

\noindent
If you have any questions please consult Jeremy Rebenstock, rebensto@msu.edu.

All the material on extra credits is at \href{{https://github.com/mhjensen/Physics321/blob/master/doc/Homeworks/ExtraCredits/}}{\nolinkurl{https://github.com/mhjensen/Physics321/blob/master/doc/Homeworks/ExtraCredits/}}. 







% ------------------- end of main content ---------------

\end{document}

