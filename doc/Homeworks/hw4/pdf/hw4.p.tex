%%
%% Automatically generated file from DocOnce source
%% (https://github.com/doconce/doconce/)
%% doconce format latex hw4.do.txt --minted_latex_style=trac --latex_admon=paragraph --no_mako
%%
% #ifdef PTEX2TEX_EXPLANATION
%%
%% The file follows the ptex2tex extended LaTeX format, see
%% ptex2tex: https://code.google.com/p/ptex2tex/
%%
%% Run
%%      ptex2tex myfile
%% or
%%      doconce ptex2tex myfile
%%
%% to turn myfile.p.tex into an ordinary LaTeX file myfile.tex.
%% (The ptex2tex program: https://code.google.com/p/ptex2tex)
%% Many preprocess options can be added to ptex2tex or doconce ptex2tex
%%
%%      ptex2tex -DMINTED myfile
%%      doconce ptex2tex myfile envir=minted
%%
%% ptex2tex will typeset code environments according to a global or local
%% .ptex2tex.cfg configure file. doconce ptex2tex will typeset code
%% according to options on the command line (just type doconce ptex2tex to
%% see examples). If doconce ptex2tex has envir=minted, it enables the
%% minted style without needing -DMINTED.
% #endif

% #define PREAMBLE

% #ifdef PREAMBLE
%-------------------- begin preamble ----------------------

\documentclass[%
oneside,                 % oneside: electronic viewing, twoside: printing
final,                   % draft: marks overfull hboxes, figures with paths
10pt]{article}

\listfiles               %  print all files needed to compile this document

\usepackage{relsize,makeidx,color,setspace,amsmath,amsfonts,amssymb}
\usepackage[table]{xcolor}
\usepackage{bm,ltablex,microtype}

\usepackage[pdftex]{graphicx}

\usepackage[T1]{fontenc}
%\usepackage[latin1]{inputenc}
\usepackage{ucs}
\usepackage[utf8x]{inputenc}

\usepackage{lmodern}         % Latin Modern fonts derived from Computer Modern

% Hyperlinks in PDF:
\definecolor{linkcolor}{rgb}{0,0,0.4}
\usepackage{hyperref}
\hypersetup{
    breaklinks=true,
    colorlinks=true,
    linkcolor=linkcolor,
    urlcolor=linkcolor,
    citecolor=black,
    filecolor=black,
    %filecolor=blue,
    pdfmenubar=true,
    pdftoolbar=true,
    bookmarksdepth=3   % Uncomment (and tweak) for PDF bookmarks with more levels than the TOC
    }
%\hyperbaseurl{}   % hyperlinks are relative to this root

\setcounter{tocdepth}{2}  % levels in table of contents

% prevent orhpans and widows
\clubpenalty = 10000
\widowpenalty = 10000

% --- end of standard preamble for documents ---


% insert custom LaTeX commands...

\raggedbottom
\makeindex
\usepackage[totoc]{idxlayout}   % for index in the toc
\usepackage[nottoc]{tocbibind}  % for references/bibliography in the toc

%-------------------- end preamble ----------------------

\begin{document}

% matching end for #ifdef PREAMBLE
% #endif

\newcommand{\exercisesection}[1]{\subsection*{#1}}


% ------------------- main content ----------------------



% ----------------- title -------------------------

\thispagestyle{empty}

\begin{center}
{\LARGE\bf
\begin{spacing}{1.25}
PHY321: Classical Mechanics 1
\end{spacing}
}
\end{center}

% ----------------- author(s) -------------------------

\begin{center}
{\bf Homework 4, due Friday  February 11${}^{}$} \\ [0mm]
\end{center}

\begin{center}
% List of all institutions:
\end{center}
    
% ----------------- end author(s) -------------------------

% --- begin date ---
\begin{center}
Feb 7, 2022
\end{center}
% --- end date ---

\vspace{1cm}


\paragraph{Practicalities about  homeworks and projects.}
\begin{enumerate}
\item You can work in groups (optimal groups are often 2-3 people) or by yourself. If you work as a group you can hand in one answer only if you wish. \textbf{Remember to write your name(s)}!

\item Homeworks are available ten days before the deadline.

\item How do I(we)  hand in?  You can hand in the paper and pencil exercises as a scanned document. For this homework this applies to exercises 1-5. Alternatively, you can hand in everyhting (if you are ok with typing mathematical formulae using say Latex) as a jupyter notebook at D2L. The numerical exercise(s) (exercise 6 here) should always be handed in as a jupyter notebook by the deadline at D2L. 
\end{enumerate}

\noindent
\paragraph{Introduction to homework 4.}
This week's sets of classical pen and paper and computational
exercises deal with simple motion problems and conservation laws;
energy, momentum and angular momentum. These conservation laws are
central in Physics and understanding them properly lays the foundation
for understanding and analyzing more complicated physics problems.

The relevant reading background is
\begin{enumerate}
\item chapters 3, 4.1, 4.2 and 4.3 of Taylor (there are many good examples there) and

\item chapters 10-13 of Malthe-Sørenssen.

\item for the numerical exercise see Malthe-Sørenssen section 7.5
\end{enumerate}

\noindent
In both textbooks there are many nice worked out examples. Malthe-Sørenssen's text contains also several coding examples you may find useful. 

The numerical homework focuses on another motion problem where you can
use the code you developed in homework 3, almost entirely. Please take
a look at the posted solution (jupyter-notebook) for homework 3. You
need only to change the forces at play. The numerical problem this time is based
on your code from homework 3 and we will try to make the motion of a falling object in two dimensions more realistic by allowing to bounce up again due to a normal force from the floor. 

\paragraph{Exercise 1 (10 pt), Conservation laws, Energy and momentum.}
\begin{itemize}
\item 1a (2pt) How do we define a conservative force?

\item 1b (4pt) Use the work-energy theorem to show that energy is conserved with a conservative force.

\item 1c (4pt) Assume that you have only internal two-body forces acting on $N$ objects in an isolated system. The force from object $i$ on object $j$ is $\bm{F}_{ij}$. Show that the linear momentum is conserved.
\end{itemize}

\noindent
\paragraph{Exercise 2 (10 pt), Conservation of angular momentum.}
\begin{itemize}
\item 2a (2pt) Define angular momentum and the torque for a single object with external forces only. 

\item 2b (4pt) Define angular momentum and the torque for a system with $N$ objects/particles  with external and internal forces. The force from object $i$ on object $j$ is $\bm{F}_{ij}$.

\item 2c (4pt) With internal forces only, what is the mathematical form of the forces that allows for angular momentum to be conserved? 
\end{itemize}

\noindent
\paragraph{Exercise 3 (10pt), Example of potential.}
Consider a particle of mass $m$ moving according to the potential
\[
V(x,y,z)=A\exp\left\{-\frac{x^2+z^2}{2a^2}\right\}.
\]

\begin{itemize}
\item 3a (2pt) Is energy conserved? If so, why? 

\item 3b (4pt) Which of  the quantities, $p_x,p_y,p_z$ are conserved?

\item 3c (4pt) Which of  the quantities, $L_x,L_y,L_z$ are conserved?
\end{itemize}

\noindent
\paragraph{Exercise 4 (15pt), Angular momentum case.}
At $t=0$ we have a single object with position $\bm{r}_0=x_0\bm{e}_x+y_0\bm{e}_y$.We add also a force in the $x$-direction at $t=0$. We assume that the object is at rest at $t=0$. 
\[
\bm{F} = F\bm{e}_x.
\]

\begin{itemize}
\item 4a (5pt) Find the velocity and momentum at a given time $t$ by integrating over time with the above initial conditions.

\item 4b (5pt) Find also the position at a time $t$. 

\item 4c (5pt) Use the position and the momentum to find the angular momentum and the torque. Is angular momentum conserved?
\end{itemize}

\noindent
\paragraph{Exercise 5 (15pt), forces  and potentials.}
A particle of mass $m$ has velocity $v=\alpha/x$, where $x$ is its displacement.

\begin{itemize}
\item 5a (5pt) Find the force $F(x)$ responsible for the motion.
\end{itemize}

\noindent
A particle is thereafter under the influence of a force $F=-kx+kx^3/\alpha^2$, where $k$ and $\alpha$ are constants and $k$ is positive.

\begin{itemize}
\item 5b (5pt) Determine the potential $U(x)$  and discuss the motion. It can be convenient here to make a sketch/plot of the potential as function of $x$.

\item 5c (5pt)  What happens when the energy of the particle is $E=(1/4)k\alpha^2$? Hint: what is the maximum value of the potential energy?
\end{itemize}

\noindent
\paragraph{Exercise 6 (40pt), Bouncing object.}
This exercise builds on the code you wrote for solving homework 3.
We recommend strongly that you study the text of Malthe-Sørenssen, section 7.5.

There we introduced gravity and air resistance and studied their
effects via a constant acceleration due to gravity and the force
arising from air resistance. But what happens when the ball hits the
floor? What if we would like to simulate the normal force from the
floor acting on the ball?  This exercise shows how we can include more
complicated forces with no pain! And the force we include here is an
example of a case where analytical solutions may either be difficult
to find or we cannot find an analytical solution at all.

We need then to include a force model for the normal force from the
floor on the ball. The simplest approach to such a system is to
introduce a contact force model represented by a spring model.  We
model the interaction between the floor and the ball as a single
spring. But the normal force is zero when there is no contact. Here we
define a simple model that allows us to include such effects in our
models.

The normal force from the floor on the ball is represented by a spring force. This
is a strong simplification of the actual deformation process occurring at the contact
between the ball and the floor due to the deformation of both the ball and the floor.

The deformed region corresponds roughly to the region of \textbf{overlap} between the
ball and the floor. The depth of this region is $\Delta y = R − y(t)$, where $R$
is the radius of the ball. This is supposed to represent the compression of the spring.
Our model for the normal force acting on the ball is then
\[
\bm{N} = −k (R − y(t)) \bm{e}_y. 
\]
The normal force must act upward when $y < R$,
hence the sign must be negative.
However, we must also ensure that the normal force only acts when the ball is in
contact with the floor, otherwise the normal force is zero. The full formation of the
normal force is therefore
\[
\bm{N} = −k (R − y(t)) \bm{e}_y, 
\]
when $y(t) < R$ and zero when $y(t) \le R$.
In the numerical calculations you can choose $R=0.1$ m and the spring constant $k=1000$ N/m.

\begin{itemize}
\item 6a (10pt) Identify the forces acting on the ball and set up a diagram with the forces acting on the ball. Find the acceleration of the falling ball now with the normal force as well.

\item 6b (30pt) Choose a large enough final time so you can study the ball bouncing up and down several times. Add the normal force and compute the height of the ball as function of time with and without air resistance. Comment your results.
\end{itemize}

\noindent
\paragraph{Classical Mechanics Extra Credit Assignment: Scientific Writing and attending Talks.}
The following gives you an opportunity to earn \textbf{five extra credit
points} on each of the remaining homeworks and \textbf{ten extra credit points}
on the midterms and finals.  This assignment also covers an aspect of
the scientific process that is not taught in most undergraduate
programs: scientific writing.  Writing scientific reports is how
scientist communicate their results to the rest of the field.  Knowing
how to assemble a well written scientific report will greatly benefit
you in you upper level classes, in graduate school, and in the work
place.

The full information on extra credits is found at \href{{https://github.com/mhjensen/Physics321/blob/master/doc/Homeworks/ExtraCredits/}}{\nolinkurl{https://github.com/mhjensen/Physics321/blob/master/doc/Homeworks/ExtraCredits/}}. There you will also find examples on how to write a scientific article. 
Below you can also find a description on how to gain extra credits by attending scientific talks.

This assignment allows you to gain extra credit points by practicing
your scientific writing.  For each of the remaining homeworks you can
submit the specified section of a scientific report (written about the
numerical aspect of the homework) for five extra credit points on the
assignment.  For the two midterms and the final, submitting a full
scientific report covering the numerical analysis problem will be
worth ten extra points.  For credit the grader must be able to tell
that you put effort into the assignment (i.e.~well written, well
formatted, etc.).  If you are unfamiliar with writing scientific
reports, \href{{https://github.com/mhjensen/Physics321/blob/master/doc/Homeworks/ExtraCredits/IntroductionScientificWriting.md}}{see the information here}

The following table explains what aspect of a scientific report is due
with which homework.  You can submit the assignment in any format you
like, in the same document as your homework, or in a different one.
Remember to cite any external references you use and include a
reference list.  There are no length requirements, but make sure what
you turn in is complete and through.  If you have any questions,
please contact Julie Butler at butler@frib.msu.edu.


\begin{quote}
\begin{tabular}{ccc}
\hline
\multicolumn{1}{c}{ HW/Project } & \multicolumn{1}{c}{ Due Date } & \multicolumn{1}{c}{ Extra Credit Assignment } \\
\hline
HW 3                       & 2-8                        & Abstract                   \\
HW 4                       & 2-15                       & Introduction               \\
HW 5                       & 2-22                       & Methods                    \\
HW 6                       & 3-1                        & Results and Discussion     \\
\textbf{Midterm 1}         & \textbf{3-12}              & \emph{Full Written Report} \\
HW 7                       & 3-22                       & Abstract                   \\
HW 8                       & 3-29                       & Introduction               \\
HW 9                       & 4-5                        & Results and Discussion     \\
\textbf{Midterm 2|} _4-16_ & \emph{Full Written Report} \\
HW 10                      & 4-26                       & Abstract                   \\
\textbf{Final}             & \textbf{4-30}              & \emph{Full Written Report} \\
\hline
\end{tabular}
\end{quote}

\noindent
You can also gain extra credits if you attend scientific talks.
This is described here.

\paragraph{Integrating Classwork With Research.}
This opportunity will allow you to earn up to 5 extra credit points on a Homework per week. These points can push you above 100\% or help make up for missed exercises.
In order to earn all points you must:

\begin{enumerate}
\item Attend an MSU research talk (recommended research oriented Clubs is  provided below)

\item Summarize the talk using at least 150 words

\item Turn in the summary along with your Homework.
\end{enumerate}

\noindent
Approved talks:
Talks given by researchers through the following clubs:
\begin{itemize}
\item Research and Idea Sharing Enterprise (RAISE)​: Meets Wednesday Nights Society for Physics Students (SPS)​: Meets Monday Nights

\item Astronomy Club​: Meets Monday Nights

\item Facility For Rare Isotope Beam (FRIB) Seminars: ​Occur multiple times a week
\end{itemize}

\noindent
If you have any questions please consult Jeremy Rebenstock, rebensto@msu.edu.

All the material on extra credits is at \href{{https://github.com/mhjensen/Physics321/blob/master/doc/Homeworks/ExtraCredits/}}{\nolinkurl{https://github.com/mhjensen/Physics321/blob/master/doc/Homeworks/ExtraCredits/}}. 

% ------------------- end of main content ---------------

% #ifdef PREAMBLE
\end{document}
% #endif

