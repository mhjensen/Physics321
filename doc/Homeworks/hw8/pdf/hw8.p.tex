%%
%% Automatically generated file from DocOnce source
%% (https://github.com/doconce/doconce/)
%% doconce format latex hw8.do.txt --no_mako
%%
% #ifdef PTEX2TEX_EXPLANATION
%%
%% The file follows the ptex2tex extended LaTeX format, see
%% ptex2tex: https://code.google.com/p/ptex2tex/
%%
%% Run
%%      ptex2tex myfile
%% or
%%      doconce ptex2tex myfile
%%
%% to turn myfile.p.tex into an ordinary LaTeX file myfile.tex.
%% (The ptex2tex program: https://code.google.com/p/ptex2tex)
%% Many preprocess options can be added to ptex2tex or doconce ptex2tex
%%
%%      ptex2tex -DMINTED myfile
%%      doconce ptex2tex myfile envir=minted
%%
%% ptex2tex will typeset code environments according to a global or local
%% .ptex2tex.cfg configure file. doconce ptex2tex will typeset code
%% according to options on the command line (just type doconce ptex2tex to
%% see examples). If doconce ptex2tex has envir=minted, it enables the
%% minted style without needing -DMINTED.
% #endif

% #define PREAMBLE

% #ifdef PREAMBLE
%-------------------- begin preamble ----------------------

\documentclass[%
oneside,                 % oneside: electronic viewing, twoside: printing
final,                   % draft: marks overfull hboxes, figures with paths
10pt]{article}

\listfiles               %  print all files needed to compile this document

\usepackage{relsize,makeidx,color,setspace,amsmath,amsfonts,amssymb}
\usepackage[table]{xcolor}
\usepackage{bm,ltablex,microtype}

\usepackage[pdftex]{graphicx}

\usepackage{ptex2tex}
% #ifdef MINTED
\usepackage{minted}
\usemintedstyle{default}
% #endif

\usepackage[T1]{fontenc}
%\usepackage[latin1]{inputenc}
\usepackage{ucs}
\usepackage[utf8x]{inputenc}

\usepackage{lmodern}         % Latin Modern fonts derived from Computer Modern

% Hyperlinks in PDF:
\definecolor{linkcolor}{rgb}{0,0,0.4}
\usepackage{hyperref}
\hypersetup{
    breaklinks=true,
    colorlinks=true,
    linkcolor=linkcolor,
    urlcolor=linkcolor,
    citecolor=black,
    filecolor=black,
    %filecolor=blue,
    pdfmenubar=true,
    pdftoolbar=true,
    bookmarksdepth=3   % Uncomment (and tweak) for PDF bookmarks with more levels than the TOC
    }
%\hyperbaseurl{}   % hyperlinks are relative to this root

\setcounter{tocdepth}{2}  % levels in table of contents

% prevent orhpans and widows
\clubpenalty = 10000
\widowpenalty = 10000

% --- end of standard preamble for documents ---


% insert custom LaTeX commands...

\raggedbottom
\makeindex
\usepackage[totoc]{idxlayout}   % for index in the toc
\usepackage[nottoc]{tocbibind}  % for references/bibliography in the toc

%-------------------- end preamble ----------------------

\begin{document}

% matching end for #ifdef PREAMBLE
% #endif

\newcommand{\exercisesection}[1]{\subsection*{#1}}


% ------------------- main content ----------------------



% ----------------- title -------------------------

\thispagestyle{empty}

\begin{center}
{\LARGE\bf
\begin{spacing}{1.25}
PHY321: Classical Mechanics 1
\end{spacing}
}
\end{center}

% ----------------- author(s) -------------------------

\begin{center}
{\bf Homework 8, due Friday March 31${}^{}$} \\ [0mm]
\end{center}

\begin{center}
% List of all institutions:
\end{center}
    
% ----------------- end author(s) -------------------------

% --- begin date ---
\begin{center}
Mar 19, 2023
\end{center}
% --- end date ---

\vspace{1cm}


\paragraph{Practicalities about  homeworks and projects.}
\begin{enumerate}
\item You can work in groups (optimal groups are often 2-3 people) or by yourself. If you work as a group you can hand in one answer only if you wish. \textbf{Remember to write your name(s)}!

\item Homeworks are available  the week before the deadline. 

\item How do I(we)  hand in?  Due to the corona virus and many of you not being on campus, we recommend that you scan your handwritten notes and upload them to D2L. If you are ok with typing mathematical formulae using say Latex, you can hand in everything as a single jupyter notebook at D2L. The numerical exercise(s) should always be handed in as a jupyter notebook by the deadline at D2L. 
\end{enumerate}

\noindent
\paragraph{Introduction to homework 8.}
This week's sets of classical pen and paper and computational
exercises are tailored to the topic of two-body problems and central
forces. It follows what was discussed during the lectures during week
12 (March 20-24) and week 12 (March 27-31).

The relevant reading background is
\begin{enumerate}
\item Sections 8.1-8.7 of Taylor.

\item Lecture notes on two-body problems, central forces and gravity.

\item For the addition exercise (optional), see \href{{https://github.com/mhjensen/Physics321/blob/master/doc/Codes/ClassicalMechanicsSolvers/WritingReusableCode.ipynb}}{Julie Butler's notes}
\end{enumerate}

\noindent
\paragraph{Exercise 1 (10pt), Equations of motion and more.}
The equations of motion in the center-of-mass frame in two dimensions with $x=r\cos{(\phi)}$ and $y=r\sin{(\phi)}$ and
$r\in [0,\infty)$, $\phi\in [0,2\pi]$ and $r=\sqrt{x^2+y^2}$ are given by
\[
\mu \ddot{r}=-\frac{dV(r)}{dr}+\mu r\dot{\phi}^2,
\]
and
\[
\dot{\phi}=\frac{L}{\mu r^2}.
\]
Here $V(r)$ is any central force which depends only on the relative coordinate.
\begin{itemize}
\item 1a (5pt) Show that you can rewrite the radial equation in terms of an effective potential $V_{\mathrm{eff}}(r)=V(r)+L^2/(2\mu r^2)$. 

\item 1b (5pt) Write out the final differential equation for the radial degrees of freedom when we specify that $V(r)=-\alpha/r$.  Plot the effective potential. You can choose values for $\alpha$ and $L$ and discuss (see Taylor section 8.4 and example 8.2) the physics of the system for two energies, one larger than zero and one smaller than zero. This is similar to what you did in the first midterm, except that the potential is different.
\end{itemize}

\noindent
\paragraph{Exercise 2 (40pt), Harmonic oscillator again.}
See the lecture notes on central forces for a discussion of this problem. It is given as an example in the lecture notes (slides from March 21-25) and the lecture text at \href{{https://mhjensen.github.io/Physics321/doc/LectureNotes/_build/html/chapter6.html#effective-or-centrifugal-potential}}{\nolinkurl{https://mhjensen.github.io/Physics321/doc/LectureNotes/_build/html/chapter6.html\#effective-or-centrifugal-potential}}.

Consider a particle of mass $m$ in a $2$-dimensional harmonic oscillator with potential
\[
V(r)=\frac{1}{2}kr^2=\frac{1}{2}k(x^2+y^2).
\]

We assume the orbit has a final non-zero angular momentum $L$.  The
effective potential looks like that of a harmonic oscillator for large
$r$, but for small $r$, the centrifugal potential repels the particle
from the origin. The combination of the two potentials has a minimum
for at some radius $r_{\rm min}$.

\begin{itemize}
\item 2a (10pt) Set up the effective potential and plot it. Find $r_{\rm min}$ and $\dot{\phi}$. Show that the latter is given by $\dot{\phi}=\sqrt{k/m}$.  At $r_{\rm min}$ the particle does not accelerate and $r$ stays constant and the motion is circular. With fixed $k$ and $m$, which parameter can we adjust to change the value of $r$ at $r_{\rm min}$?

\item 2b (10pt) Now consider small vibrations about $r_{\rm min}$. The effective spring constant is the curvature of the effective potential.  Use the curvature at $r_{\rm min}$ to find the effective spring constant (hint, look at  exercise 4 in homework 6) $k_{\mathrm{eff}}$. Show also that $\omega=\sqrt{k_{\mathrm{eff}}/m}=2\dot{\phi}$  

\item 2c (10pt) The solution to the equations of motion in Cartesian coordinates is simple. The $x$ and $y$ equations of motion separate, and we have $\ddot{x}=-kx/m$ and $\ddot{y}=-ky/m$. The harmonic oscillator is indeed a system where the degrees of freedom separate and we can find analytical solutions. Define a natural frequency $\omega_0=\sqrt{k/m}$ and show that (where $A$, $B$, $C$ and $D$ are arbitrary constants defined by the initial conditions)
\end{itemize}

\noindent
\begin{eqnarray*}
x&=&A\cos\omega_0 t+B\sin\omega_0 t,\\
y&=&C\cos\omega_0 t+D\sin\omega_0 t.
\end{eqnarray*}

\begin{itemize}
\item 2d (10pt) With the solutions for $x$ and $y$, and $r^2=x^2+y^2$ and the definitions $\alpha=\frac{A^2+B^2+C^2+D^2}{2}$, $\beta=\frac{A^2-B^2+C^2-D^2}{2}$ and $\gamma=AB+CD$, show that
\end{itemize}

\noindent
\[
r^2=\alpha+(\beta^2+\gamma^2)^{1/2}\cos(2\omega_0 t-\delta),
\]
with
\[
\delta=\arctan(\gamma/\beta),
\]

\paragraph{Exercise 3 (40pt), Numerical Solution of the Harmonic Oscillator.}
Using the code we developed in homework 5 or the first midterm (part b), we can solve the above harmonic oscillator problem in two dimensions using our code from this homework. We need however to change the acceleration from the gravitational force to the one given by the harmonic oscillator potential.

The code is given here for the Velocity-Verlet algorithm, with obvious elements to fill in. 



































































\bpycod
# Common imports
import numpy as np
import pandas as pd
from math import *
import matplotlib.pyplot as plt
import os

# Where to save the figures and data files
PROJECT_ROOT_DIR = "Results"
FIGURE_ID = "Results/FigureFiles"
DATA_ID = "DataFiles/"

if not os.path.exists(PROJECT_ROOT_DIR):
    os.mkdir(PROJECT_ROOT_DIR)

if not os.path.exists(FIGURE_ID):
    os.makedirs(FIGURE_ID)

if not os.path.exists(DATA_ID):
    os.makedirs(DATA_ID)

def image_path(fig_id):
    return os.path.join(FIGURE_ID, fig_id)

def data_path(dat_id):
    return os.path.join(DATA_ID, dat_id)

def save_fig(fig_id):
    plt.savefig(image_path(fig_id) + ".png", format='png')

DeltaT = 0.01
#set up arrays 
tfinal = 10.0
n = ceil(tfinal/DeltaT)
# set up arrays for t, a, v, and x
t = np.zeros(n)
v = np.zeros((n,2))
r = np.zeros((n,2))
# Initial conditions as compact 2-dimensional arrays
r0 = np.array([1.0,0.5])  # You need to  change these to fit rmin
v0 = np.array([0.0,0.0]) # You need to change these to fit rmin
r[0] = r0
v[0] = v0
k = 1.0   # spring constant
m = 1.0   # mass, you can change these
omega02 = k/m
# Start integrating using the Velocity-Verlet  method
for i in range(n-1):
    # Set up forces, define acceleration first
    a =  -r[i]*omega02  # you may need to change this
    # update velocity, time and position using the Velocity-Verlet method
    r[i+1] = r[i] + DeltaT*v[i]+0.5*(DeltaT**2)*a
    # new accelerationfor the Verlet method
    anew = -r[i+1]*omega02  
    v[i+1] = v[i] + 0.5*DeltaT*(a+anew)
    t[i+1] = t[i] + DeltaT
# Plot position as function of time    
fig, ax = plt.subplots()
ax.set_xlabel('t[s]')
ax.set_ylabel('x[m] and y[m]')
ax.plot(t,r[:,0])
ax.plot(t,r[:,1])
fig.tight_layout()
save_fig("2DimHOVV")
plt.show()


\epycod


\begin{itemize}
\item 3a (20pt) Use for example the above code to set up the acceleration and use the initial conditions fixed by for example $r_{\rm min}$ from exercise 2. Which value should the initial velocity take if you place yourself at $r_{\rm min}$ and you require a circular motion? Hint: see the first midterm, part 2. There you used the centripetal acceleration.  
\end{itemize}

\noindent
Instead of solving the equations in the cartesian frame we will now rewrite the above code in terms of the radial degrees of freedom only. Our differential equation is now
\[
\mu \ddot{r}=-\frac{dV(r)}{dr}+\mu r\dot{\phi}^2,
\]
and
\[
\dot{\phi}=\frac{L}{\mu r^2}.
\]

\begin{itemize}
\item 3b (20pt) We will use $r_{\rm min}$ to fix a value of $L$, as seen in exercise 2. This fixes also $\dot{\phi}$. Write a code which now implements the radial equation for $r$ using the same $r_{\rm min}$ as you did in 3a. Compare the results with those from 3a with the same initial conditions. Do they agree? Use only one set of initial conditions.
\end{itemize}

\noindent
\paragraph{Exercise 4, equations for an ellipse (10pt).}
Consider an ellipse defined by the sum of the distances from the two foci being $2D$, which expressed in a Cartesian coordinates with the middle of the ellipse being at the origin becomes
\[
\sqrt{(x-a)^2+y^2}+\sqrt{(x+a)^2+y^2}=2D.
\]
Here the two foci are at $(a,0)$ and $(-a,0)$. Show that this form is can be written as
\[
\frac{x^2}{D^2}+\frac{y^2}{D^2-a^2}=1.
\]

\paragraph{Bonus exercise, from \href{{https://github.com/mhjensen/Physics321/blob/master/doc/Codes/ClassicalMechanicsSolvers/WritingReusableCode.ipynb}}{Julie Butler's notes}.}
This bonus exercise gives you an additional score of \textbf{20pt}.

Make a class called Solvers which contains the following:
\begin{enumerate}
\item An initializer

\item A method that implements Euler's method in 2D

\item A method that implements Euler-Cromer's method in 2D

\item A method that implements Velocity-Verlet in 2D

\item A method that plots position versus time results from the solver
\end{enumerate}

\noindent
oAt least three class level variables

The class may also contain any other methods or class level variables that you think are needed.

Make then a function (outside of the class Solver) that solves the Earth-Sun problem using the Solvers class (Exercise 6 from Homework 5).

\paragraph{Classical Mechanics Extra Credit Assignment: Scientific Writing and attending Talks.}
The following gives you an opportunity to earn \textbf{five extra credit
points} on each of the remaining homeworks and \textbf{ten extra credit points}
on the midterms and finals.  This assignment also covers an aspect of
the scientific process that is not taught in most undergraduate
programs: scientific writing.  Writing scientific reports is how
scientist communicate their results to the rest of the field.  Knowing
how to assemble a well written scientific report will greatly benefit
you in you upper level classes, in graduate school, and in the work
place.

The full information on extra credits is found at \href{{https://github.com/mhjensen/Physics321/blob/master/doc/Homeworks/ExtraCredits/}}{\nolinkurl{https://github.com/mhjensen/Physics321/blob/master/doc/Homeworks/ExtraCredits/}}. There you will also find examples on how to write a scientific article. 
Below you can also find a description on how to gain extra credits by attending scientific talks.

This assignment allows you to gain extra credit points by practicing
your scientific writing.  For each of the remaining homeworks you can
submit the specified section of a scientific report (written about the
numerical aspect of the homework) for five extra credit points on the
assignment.  For the two midterms and the final, submitting a full
scientific report covering the numerical analysis problem will be
worth ten extra points.  For credit the grader must be able to tell
that you put effort into the assignment (i.e.~well written, well
formatted, etc.).  If you are unfamiliar with writing scientific
reports, \href{{https://github.com/mhjensen/Physics321/blob/master/doc/Homeworks/ExtraCredits/IntroductionScientificWriting.md}}{see the information here}

The following table explains what aspect of a scientific report is due
with which homework.  You can submit the assignment in any format you
like, in the same document as your homework, or in a different one.
Remember to cite any external references you use and include a
reference list.  There are no length requirements, but make sure what
you turn in is complete and through.  If you have any questions,
please contact Julie Butler at butler@frib.msu.edu.


\begin{quote}
\begin{tabular}{ccc}
\hline
\multicolumn{1}{c}{ HW/Project } & \multicolumn{1}{c}{ Due Date } & \multicolumn{1}{c}{ Extra Credit Assignment } \\
\hline
HW 3               & 2-8           & Abstract                   \\
HW 4               & 2-15          & Introduction               \\
HW 5               & 2-22          & Methods                    \\
HW 6               & 3-1           & Results and Discussion     \\
\textbf{Midterm 1} & \textbf{3-12} & \emph{Full Written Report} \\
HW 7               & 3-22          & Abstract                   \\
HW 8               & 3-29          & Introduction               \\
HW 9               & 4-5           & Results and Discussion     \\
\textbf{Midterm 2} & \textbf{4-16} & \emph{Full Written Report} \\
HW 10              & 4-26          & Abstract                   \\
\textbf{Final}     & \textbf{4-30} & \emph{Full Written Report} \\
\hline
\end{tabular}
\end{quote}

\noindent
You can also gain extra credits if you attend scientific talks.
This is described here.

\paragraph{Integrating Classwork With Research.}
This opportunity will allow you to earn up to 5 extra credit points on a Homework per week. These points can push you above 100\% or help make up for missed exercises.
In order to earn all points you must:

\begin{enumerate}
\item Attend an MSU research talk (recommended research oriented Clubs is  provided below)

\item Summarize the talk using at least 150 words

\item Turn in the summary along with your Homework.
\end{enumerate}

\noindent
Approved talks:
Talks given by researchers through the following clubs:
\begin{itemize}
\item Research and Idea Sharing Enterprise (RAISE)​: Meets Wednesday Nights Society for Physics Students (SPS)​: Meets Monday Nights

\item Astronomy Club​: Meets Monday Nights

\item Facility For Rare Isotope Beam (FRIB) Seminars: ​Occur multiple times a week
\end{itemize}

\noindent
All the material on extra credits is at \href{{https://github.com/mhjensen/Physics321/blob/master/doc/Homeworks/ExtraCredits/}}{\nolinkurl{https://github.com/mhjensen/Physics321/blob/master/doc/Homeworks/ExtraCredits/}}. 


% ------------------- end of main content ---------------

% #ifdef PREAMBLE
\end{document}
% #endif

