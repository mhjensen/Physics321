%%
%% Automatically generated file from DocOnce source
%% (https://github.com/hplgit/doconce/)
%%
%%
% #ifdef PTEX2TEX_EXPLANATION
%%
%% The file follows the ptex2tex extended LaTeX format, see
%% ptex2tex: https://code.google.com/p/ptex2tex/
%%
%% Run
%%      ptex2tex myfile
%% or
%%      doconce ptex2tex myfile
%%
%% to turn myfile.p.tex into an ordinary LaTeX file myfile.tex.
%% (The ptex2tex program: https://code.google.com/p/ptex2tex)
%% Many preprocess options can be added to ptex2tex or doconce ptex2tex
%%
%%      ptex2tex -DMINTED myfile
%%      doconce ptex2tex myfile envir=minted
%%
%% ptex2tex will typeset code environments according to a global or local
%% .ptex2tex.cfg configure file. doconce ptex2tex will typeset code
%% according to options on the command line (just type doconce ptex2tex to
%% see examples). If doconce ptex2tex has envir=minted, it enables the
%% minted style without needing -DMINTED.
% #endif

% #define PREAMBLE

% #ifdef PREAMBLE
%-------------------- begin preamble ----------------------

\documentclass[%
oneside,                 % oneside: electronic viewing, twoside: printing
final,                   % draft: marks overfull hboxes, figures with paths
10pt]{article}

\listfiles               %  print all files needed to compile this document

\usepackage{relsize,makeidx,color,setspace,amsmath,amsfonts,amssymb}
\usepackage[table]{xcolor}
\usepackage{bm,ltablex,microtype}

\usepackage[pdftex]{graphicx}

\usepackage[T1]{fontenc}
%\usepackage[latin1]{inputenc}
\usepackage{ucs}
\usepackage[utf8x]{inputenc}

\usepackage{lmodern}         % Latin Modern fonts derived from Computer Modern

% Hyperlinks in PDF:
\definecolor{linkcolor}{rgb}{0,0,0.4}
\usepackage{hyperref}
\hypersetup{
    breaklinks=true,
    colorlinks=true,
    linkcolor=linkcolor,
    urlcolor=linkcolor,
    citecolor=black,
    filecolor=black,
    %filecolor=blue,
    pdfmenubar=true,
    pdftoolbar=true,
    bookmarksdepth=3   % Uncomment (and tweak) for PDF bookmarks with more levels than the TOC
    }
%\hyperbaseurl{}   % hyperlinks are relative to this root

\setcounter{tocdepth}{2}  % levels in table of contents

% prevent orhpans and widows
\clubpenalty = 10000
\widowpenalty = 10000

% --- end of standard preamble for documents ---


% insert custom LaTeX commands...

\raggedbottom
\makeindex
\usepackage[totoc]{idxlayout}   % for index in the toc
\usepackage[nottoc]{tocbibind}  % for references/bibliography in the toc

%-------------------- end preamble ----------------------

\begin{document}

% matching end for #ifdef PREAMBLE
% #endif

\newcommand{\exercisesection}[1]{\subsection*{#1}}


% ------------------- main content ----------------------



% ----------------- title -------------------------

\thispagestyle{empty}

\begin{center}
{\LARGE\bf
\begin{spacing}{1.25}
PHY321: Classical Mechanics 1
\end{spacing}
}
\end{center}

% ----------------- author(s) -------------------------

\begin{center}
{\bf Second midterm project, due Friday April 16${}^{}$} \\ [0mm]
\end{center}

\begin{center}
% List of all institutions:
\end{center}
    
% ----------------- end author(s) -------------------------

% --- begin date ---
\begin{center}
Apr 1, 2021
\end{center}
% --- end date ---

\vspace{1cm}


\paragraph{Practicalities about  homeworks and projects.}
\begin{enumerate}
\item You can work in groups (optimal groups are often 2-3 people) or by yourself. If you work as a group you can hand in one answer only if you wish. \textbf{Remember to write your name(s)}!

\item How do I(we)  hand in?  Due to the extraordinary situation we are in now, the midterm should be handed in fully via D2L. You can scan your handwritten notes and upload to D2L or you can hand in everyhting (if you are ok with typing mathematical formulae using say Latex) as a jupyter notebook at D2L. The numerical part should always be handed in as a jupyter notebook.
\end{enumerate}

\noindent
\paragraph{Introduction to the second midterm project, total score 100 points.}
In this midterm we will attempt at writing a program that simulates
the solar system. We start with the Earth-Sun system we studied in
homework 5 and 6 and study elliptical orbits and their properties. We test
also elliptical orbits and their dependence on powers $\beta$ of
$r^{\beta}$. We will test other aspects of the Earth-Sun system and
link these to the theoretical discussion on two-body problems with
central forces. 

Thereafter, based on the three-body problem studied in homework 9, we
attempt at making a code which simulates the solar system.

The relevant reading background is
\begin{enumerate}
\item chapter 8 of  Taylor.

\item Lecture notes on central forces and two-body problems

\item Homeworks 5-9
\end{enumerate}

\noindent
\paragraph{Part 1 (50pt), the inverse-square law and the stability of planetary orbits.}
In homework 9 we studied an attractive potential
\[
V(r)=-\alpha/r,
\]

where the quantity $r$ is the absolute value of the relative position and $\alpha$ is a constant.

When we rewrote the equations of motion in polar coordinates, the analytical solution to the radial equation of motion was
\[
r(\phi) = \frac{c}{1+\epsilon\cos{(\phi)}},
\]
where $c=L^2/\mu\alpha$, with
the reduced mass $\mu$ and the angular momentum $L$, as
discussed during the lectures. With the transformation of a two-body
problem to the center-of-mass frame, the actual equations look like an
\emph{effective} one-body problem. 

The quantity $\epsilon$ is what we called the eccentricity. Since we will mainly study bounded orbits,
we have $0 \le \epsilon < 1$.
For the Earth, the orbit is indeed close to circular and at perihelion (the closest distance to the Sun), the Earth's center is about 0.98329 astronomical units (AU) or 147,098,070 km from the Sun's center. For Earth, the orbital eccentricity is $\epsilon\approx 0.0167$. The outer planets have more elliptical orbits. For example, Mars has its perihelion at 206,655,215 km and its apehelion at 249,232,432 km. 

In this part we will limit ourselves to the Earth-Sun system we studied in homeworks 5 and 6. You can reuse your code with either the Velocity-Verlet or the Euler-Cromer algorithms from homework 5 or 6.

This means also that $\alpha=GM_{\odot}M_{\mathrm{Earth}}$. We will
use $\alpha$ as a shorthand in the equations here. Keep in mind that
in homework 5 you scaled $GM_{\odot}=4\pi^2$ in your code.

The exercises here are all based on you analyzing the results from your code from homeworks 5, 6, 7 and 8.


\begin{itemize}
\item 1a (10pt) Use now your code from homework 5 (in cartesian coordinates). Start with a circular orbit setting $\epsilon=0$ and plot $x$ versus $y$. How would you choose the initial conditions to obtain a circular orbit?

\item 1b (10pt) Check that for the case of a circular orbit that both the kinetic and the potential energies are conserved. Why do we expect such a result if we have a circular orbit? 

\item 1c (10pt) With the same initial conditions (circular orbit) Use Kepler's second law (see Taylor section 3.4) to show that angular momentum is conserved. Compare the value you get with the angular momentum you get from a circular orbit. 

\item 1d (10pt) Till now we have assumed that we have an inverse-square force $F(r) = -\alpha/r^2$. Let us rewrite this force as $F(r) = -\alpha/r^{\beta}$ with $\beta=[2,2.01,2.10,2.5,3.0,3.5]$. Run your Sun-Earth code with these values of $\beta$ and plot $x$ versus $y$ (you can use the same initial conditions or switch to eliptical orbits). Discuss your results. Can you use the observations of planetary motion to determine by what amount Nature deviates from a perfect inverse-square law? 

\item 1e (10pt) Consider now an elliptical orbit with an initial position 1 AU from the Sun and an initial  velocity of 5 AU/yr. Show that the total energy is a constant (the kinetic and potential energies will vary). Show also that the angular momentum is a constant. If you change the parameter $\beta$ in $F(r) = -\alpha/r^{\beta}$ from $\beta=2$ to $\beta=3$, are these quantities conserved?  Discuss your results. (Hint: relate your results to Kepler's laws). 
\end{itemize}

\noindent
\paragraph{Part 2 (50pt), making a program for the solar system.}
Our final aim is to write a code which includes the known planets of the solar system. 

We will, as before, use so-called astronomical units when rewriting our equations. 
Using astronomical units (AU as abbreviation)it means that 
one astronomical unit of length, known as 1 AU, is the average distance between the Sun and Earth, that is
$1$ AU = $1.5\times 10^{11}$ m.  It can also be convenient to use years instead of seconds since years match
better the time evolution of the solar system. The mass of the Sun is $M_{\mathrm{sun}}=M_{\odot}=2\times 10^{30}$ kg. The masses of all relevant planets and their distances from the sun are listed in the table here in kg and AU.


\begin{quote}
\begin{tabular}{ccc}
\hline
\multicolumn{1}{c}{ Planet } & \multicolumn{1}{c}{ Mass in kg } & \multicolumn{1}{c}{ Distance to  sun in AU } \\
\hline
Earth   & $M_{\mathrm{Earth}}=6\times 10^{24}$ kg     & 1AU                    \\
Jupiter & $M_{\mathrm{Jupiter}}=1.9\times 10^{27}$ kg & 5.20 AU                \\
Mars    & $M_{\mathrm{Mars}}=6.6\times 10^{23}$ kg    & 1.52 AU                \\
Venus   & $M_{\mathrm{Venus}}=4.9\times 10^{24}$ kg   & 0.72 AU                \\
Saturn  & $M_{\mathrm{Saturn}}=5.5\times 10^{26}$ kg  & 9.54 AU                \\
Mercury & $M_{\mathrm{Mercury}}=3.3\times 10^{23}$ kg & 0.39 AU                \\
Uranus  & $M_{\mathrm{Uranus}}=8.8\times 10^{25}$ kg  & 19.19 AU               \\
Neptun  & $M_{\mathrm{Neptun}}=1.03\times 10^{26}$ kg & 30.06 AU               \\
Pluto   & $M_{\mathrm{Pluto}}=1.31\times 10^{22}$ kg  & 39.53 AU               \\
\hline
\end{tabular}
\end{quote}

\noindent
Pluto is no longer considered  a planet, but we add it here for historical reasons. It is optional in this midterm project to include Pluto and eventual moons. 

In setting up the equations we can limit ourselves to a co-planar motion and use only the $x$ and $y$ coordinates. But you should feel free to extend your equations to three dimensions, it is not very difficult and the data from NASA are all in three dimensions.

\href{{http://www.nasa.gov/index.html}}{NASA} has an excellent site at \href{{http://ssd.jpl.nasa.gov/horizons.cgi#top}}{\nolinkurl{http://ssd.jpl.nasa.gov/horizons.cgi\#top}}.
From there you can extract initial conditions in order to start your differential equation solver.
At the above website you need to change from \textbf{OBSERVER} to \textbf{VECTOR} and then write in the planet you are interested in.
The generated data contain the $x$, $y$ and $z$ values as well as their corresponding velocities. The velocities are in units of AU per day.
Alternatively they can be obtained in terms of km and km/s. 


\begin{itemize}
\item 2a (50pt) Since the Sun is much more massive than all the other planets, we will define the Sun as our center of mass and set its velocity and position to zero. You can use your code from homework 9 and add gradually one planet at the time. Develop a code which simulates the solar system with the above planets and plot their orbits. Discuss your results.
\end{itemize}

\noindent
In homework 4 we limited ourselves (in order to test the algorithm) to
a hypothetical solar system with the Earth only orbiting around the
Sun. We assumed that the only force in the problem is
gravity. Newton's law of gravitation is given by a force $F_G$ (we assume this is the force acting on Earth from the Sun)

\[
F_G=-\frac{GM_{\odot}M_{\mathrm{Earth}}}{r^2},
\]

where $M_{\odot}$ is the mass of the Sun and $M_{\mathrm{Earth}}$ is
the mass of the Earth. The gravitational constant is $G$ and $r$ is
the distance between the Earth and the Sun.  We assumed that the Sun
has a mass which is much larger than that of the Earth. We could
therefore safely neglect the motion of the Sun.

In homework 4 assumed that the orbit of the Earth around the Sun 
was co-planar, and we took this to be the $xy$-plane.
Using Newton's second law of motion we got the following equations

\[
\frac{d^2x}{dt^2}=-\frac{F_{G,x}}{M_{\mathrm{Earth}}},
\]
and

\[
\frac{d^2y}{dt^2}=-\frac{F_{G,y}}{M_{\mathrm{Earth}}},
\]

where $F_{G,x}$ and $F_{G,y}$ are the $x$ and $y$ components of the
gravitational force.

We will again use so-called astronomical units when rewriting our
equations.  Using astronomical units (AU as abbreviation)it means that
one astronomical unit of length, known as 1 AU, is the average
distance between the Sun and Earth, that is $1$ AU = $1.5\times
10^{11}$ m.  It can also be convenient to use years instead of seconds
since years match better the time evolution of the solar system. The
mass of the Sun is $M_{\mathrm{sun}}=M_{\odot}=2\times 10^{30}$
kg. The masses of all relevant planets and their distances from the
sun are listed in the table here in kg and AU.


\begin{quote}
\begin{tabular}{ccc}
\hline
\multicolumn{1}{c}{ Planet } & \multicolumn{1}{c}{ Mass in kg } & \multicolumn{1}{c}{ Distance to  sun in AU } \\
\hline
Earth   & $M_{\mathrm{Earth}}=6\times 10^{24}$ kg     & 1AU                    \\
Jupiter & $M_{\mathrm{Jupiter}}=1.9\times 10^{27}$ kg & 5.20 AU                \\
Mars    & $M_{\mathrm{Mars}}=6.6\times 10^{23}$ kg    & 1.52 AU                \\
Venus   & $M_{\mathrm{Venus}}=4.9\times 10^{24}$ kg   & 0.72 AU                \\
Saturn  & $M_{\mathrm{Saturn}}=5.5\times 10^{26}$ kg  & 9.54 AU                \\
Mercury & $M_{\mathrm{Mercury}}=3.3\times 10^{23}$ kg & 0.39 AU                \\
Uranus  & $M_{\mathrm{Uranus}}=8.8\times 10^{25}$ kg  & 19.19 AU               \\
Neptun  & $M_{\mathrm{Neptun}}=1.03\times 10^{26}$ kg & 30.06 AU               \\
Pluto   & $M_{\mathrm{Pluto}}=1.31\times 10^{22}$ kg  & 39.53 AU               \\
\hline
\end{tabular}
\end{quote}

\noindent
Pluto is no longer considered a planet, but we add it here for
historical reasons. It is optional in this project to include Pluto
and eventual moons.

In setting up the equations we can limit ourselves to a co-planar
motion and use only the $x$ and $y$ coordinates. But you should feel
free to extend your equations to three dimensions, it is not very
difficult and the data from NASA are all in three dimensions.

\href{{http://www.nasa.gov/index.html}}{NASA} has an excellent site at \href{{http://ssd.jpl.nasa.gov/horizons.cgi#top}}{\nolinkurl{http://ssd.jpl.nasa.gov/horizons.cgi\#top}}.
From there you can extract initial conditions in order to start your differential equation solver.
At the above website you need to change from \textbf{OBSERVER} to \textbf{VECTOR} and then write in the planet you are interested in.
The generated data contain the $x$, $y$ and $z$ values as well as their corresponding velocities. The velocities are in units of AU per day.
Alternatively they can be obtained in terms of km and km/s. 


Using our code from homework 4, we will now add Jupyter and play
around with different masses for this planet and study numerically a
three-body problem.  This is a well-studied problem in classical
mechanics, \href{{https://en.wikipedia.org/wiki/Three-body_problem}}{with many interesting results, from stable orbits to chaotic motion}.

\paragraph{Exercise 1 The three-body problem (100pt).}
We will now study the three-body problem, still with the Sun kept
fixed as the center of mass of the system but including Jupiter (the
most massive planet in the solar system, having a mass that is
approximately 1000 times smaller than that of the Sun) together with
the Earth. This leads to a three-body problem. Without Jupiter, the
Earth's motion is stable and unchanging with time. The aim here is to
find out how much Jupiter alters the Earth's motion.

The program you have developed in homework 4 can easily be modified by
simply adding the magnitude of the force betweem the Earth and
Jupiter.

This force is given again by

\[
F_{\mathrm{Earth-Jupiter}}=-\frac{GM_{\mathrm{Jupiter}}M_{\mathrm{Earth}}}{r_{\mathrm{Earth-Jupiter}}^2},
\]

where $M_{\mathrm{Jupiter}}$ is the mass of Jupyter and
$M_{\mathrm{Earth}}$ is the mass of the Earth.  The gravitational constant
is $G$ and $r_{\mathrm{Earth-Jupiter}}$ is the distance between Earth
and Jupiter.

We assume again that the orbits of the two planets are co-planar, and
we take this to be the $xy$-plane (you can easily extend the equations
to three dimensions).

\begin{itemize}
\item 1a (20pt) Modify your coupled first-order differential equations from homework 4 in order to accomodate both the motion of the Earth and Jupiter by taking into account the distance in $x$ and $y$ between the Earth and Jupiter. Write out the differential equations for  Earth and Jupyter, keeping the Sun at rest (mass center of the system).

\item 1b (10pt) Scale these equations in terms of Astronomical Units.

\item 1c (30pt) Set up the algorithm and plot the positions of the Earth and Jupiter using the Velocity Verlet algorithm. Discuss the stability of the solutions using your Velocity Verlet solver.

\item 1c (40pt)  Repeat the calculations by increasing the mass of Jupiter by a factor of 10, 100 and 1000 and plot the position of the Earth.  Discuss your results and study again the stability of the Velocity Verlet solver. Is energy conserved? 
\end{itemize}

\noindent
\paragraph{Exercise 2, the bonus part (50pt).  The perihelion precession of Mercury.}
This is the bonus exercise and gives an additional score of 50
points. It is fully optional. \textbf{I would grade this as a more difficult
exercise compared to previous ones}. \href{{https://en.wikipedia.org/wiki/Tests_of_general_relativity}}{It requires also that you read
some background literature about the perihelion of Mercury}. You don't
need to derive the relativistic correction here. This is something you
will meet in a graduate course on General Relativity. The bonus here
is that it allows you explore physics you could not have done without
a numerical code.

An important test of the general theory of relativity was to compare
its prediction for the perihelion precession of Mercury to the
observed value. The observed value of the perihelion precession, when
all classical effects (such as the perturbation of the orbit due to
gravitational attraction from the other planets) are subtracted, is
$43''$ ($43$ arc seconds) per century.

Closed elliptical orbits are a special feature of the Newtonian
$1/r^2$ force. In general, any correction to the pure $1/r^2$
behaviour will lead to an orbit which is not closed, i.e.~after one
complete orbit around the Sun, the planet will not be at exactly the
same position as it started. If the correction is small, then each
orbit around the Sun will be almost the same as the classical ellipse,
and the orbit can be thought of as an ellipse whose orientation in
space slowly rotates. In other words, the perihelion of the ellipse
slowly precesses around the Sun.

You will now study the orbit of Mercury around the Sun, adding a general relativistic correction to the Newtonian
gravitational force, so that the force becomes

\[
F = -\frac{GM_\mathrm{Sun}M_\mathrm{Mercury}}{r^2}\left[1 + \frac{3l^2}{r^2c^2}\right]
\]

where $M_\mathrm{Mercury}$ is the mass of Mercury, $r$ is the distance
between Mercury and the Sun, $l=|\vec{r}\times\vec{v}|$ is the
magnitude of Mercury's orbital angular momentum per unit mass, and $c$
is the speed of light in vacuum. Run a simulation over one century of
Mercury's orbit around the Sun with no other planets present, starting
with Mercury at the perihelion on the $x$ axis.  Check then the value
of the perihelion angle $\theta_\mathrm{p}$, using

\[
\tan \theta_\mathrm{p} = \frac{y_\mathrm{p}}{x_\mathrm{p}}
\]

where $x_\mathrm{p}$ ($y_\mathrm{p}$) is the $x$ ($y$) position of
Mercury at perihelion, i.e.~at the point where Mercury is at its
closest to the Sun. You may use that the speed of Mercury at
perihelion is $12.44\,\mathrm{AU}/\mathrm{yr}$, and that the distance
to the Sun at perihelion is $0.3075\,\mathrm{AU}$.

You need to make
sure that the time resolution used in your simulation is sufficient,
for example by checking that the perihelion precession you get with a
pure Newtonian force is at least a few orders of magnitude smaller
than the observed perihelion precession of Mercury. Can the observed
perihelion precession of Mercury be explained by the general theory of
relativity?


% ------------------- end of main content ---------------

% #ifdef PREAMBLE
\end{document}
% #endif

