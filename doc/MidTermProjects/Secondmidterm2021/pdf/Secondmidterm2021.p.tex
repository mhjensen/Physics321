%%
%% Automatically generated file from DocOnce source
%% (https://github.com/hplgit/doconce/)
%%
%%
% #ifdef PTEX2TEX_EXPLANATION
%%
%% The file follows the ptex2tex extended LaTeX format, see
%% ptex2tex: https://code.google.com/p/ptex2tex/
%%
%% Run
%%      ptex2tex myfile
%% or
%%      doconce ptex2tex myfile
%%
%% to turn myfile.p.tex into an ordinary LaTeX file myfile.tex.
%% (The ptex2tex program: https://code.google.com/p/ptex2tex)
%% Many preprocess options can be added to ptex2tex or doconce ptex2tex
%%
%%      ptex2tex -DMINTED myfile
%%      doconce ptex2tex myfile envir=minted
%%
%% ptex2tex will typeset code environments according to a global or local
%% .ptex2tex.cfg configure file. doconce ptex2tex will typeset code
%% according to options on the command line (just type doconce ptex2tex to
%% see examples). If doconce ptex2tex has envir=minted, it enables the
%% minted style without needing -DMINTED.
% #endif

% #define PREAMBLE

% #ifdef PREAMBLE
%-------------------- begin preamble ----------------------

\documentclass[%
oneside,                 % oneside: electronic viewing, twoside: printing
final,                   % draft: marks overfull hboxes, figures with paths
10pt]{article}

\listfiles               %  print all files needed to compile this document

\usepackage{relsize,makeidx,color,setspace,amsmath,amsfonts,amssymb}
\usepackage[table]{xcolor}
\usepackage{bm,ltablex,microtype}

\usepackage[pdftex]{graphicx}

\usepackage[T1]{fontenc}
%\usepackage[latin1]{inputenc}
\usepackage{ucs}
\usepackage[utf8x]{inputenc}

\usepackage{lmodern}         % Latin Modern fonts derived from Computer Modern

% Hyperlinks in PDF:
\definecolor{linkcolor}{rgb}{0,0,0.4}
\usepackage{hyperref}
\hypersetup{
    breaklinks=true,
    colorlinks=true,
    linkcolor=linkcolor,
    urlcolor=linkcolor,
    citecolor=black,
    filecolor=black,
    %filecolor=blue,
    pdfmenubar=true,
    pdftoolbar=true,
    bookmarksdepth=3   % Uncomment (and tweak) for PDF bookmarks with more levels than the TOC
    }
%\hyperbaseurl{}   % hyperlinks are relative to this root

\setcounter{tocdepth}{2}  % levels in table of contents

% prevent orhpans and widows
\clubpenalty = 10000
\widowpenalty = 10000

% --- end of standard preamble for documents ---


% insert custom LaTeX commands...

\raggedbottom
\makeindex
\usepackage[totoc]{idxlayout}   % for index in the toc
\usepackage[nottoc]{tocbibind}  % for references/bibliography in the toc

%-------------------- end preamble ----------------------

\begin{document}

% matching end for #ifdef PREAMBLE
% #endif

\newcommand{\exercisesection}[1]{\subsection*{#1}}


% ------------------- main content ----------------------



% ----------------- title -------------------------

\thispagestyle{empty}

\begin{center}
{\LARGE\bf
\begin{spacing}{1.25}
PHY321: Classical Mechanics 1
\end{spacing}
}
\end{center}

% ----------------- author(s) -------------------------

\begin{center}
{\bf Second midterm project, due Friday April 16${}^{}$} \\ [0mm]
\end{center}

\begin{center}
% List of all institutions:
\end{center}
    
% ----------------- end author(s) -------------------------

% --- begin date ---
\begin{center}
Apr 3, 2021
\end{center}
% --- end date ---

\vspace{1cm}


\paragraph{Practicalities about  homeworks and projects.}
\begin{enumerate}
\item You can work in groups (optimal groups are often 2-3 people) or by yourself. If you work as a group you can hand in one answer only if you wish. \textbf{Remember to write your name(s)}!

\item How do I(we)  hand in?  Due to the extraordinary situation we are in now, the midterm should be handed in fully via D2L. You can scan your handwritten notes and upload to D2L or you can hand in everything (if you are ok with typing mathematical formulae using say Latex) as a jupyter notebook at D2L. The numerical part should always be handed in as a jupyter notebook.
\end{enumerate}

\noindent
\paragraph{Introduction to the second midterm project, total score 100 points.}
In this midterm we will attempt at writing a program that simulates
the solar system. We start with the Earth-Sun system we studied in
homeworks 5 and 6 and study elliptical orbits and their properties. We test
also elliptical orbits and their dependence on powers $\beta$ of
$r^{\beta}$. We will test other aspects of the Earth-Sun system and
link these to the theoretical discussion on two-body problems with
central forces. 

Thereafter we will add Jupiter to our system before we move on to including all planets of the solar system.
attempt at making a code which simulates the solar system.

The relevant reading background is
\begin{enumerate}
\item chapter 8 of  Taylor.

\item Lecture notes on central forces and two-body problems, see also the Jupyter-Book at \href{{https://mhjensen.github.io/Physics321/doc/LectureNotes/_build/html/}}{\nolinkurl{https://mhjensen.github.io/Physics321/doc/LectureNotes/_build/html/}} and go \href{{https://mhjensen.github.io/Physics321/doc/LectureNotes/_build/html/chapter6.html}}{Two-body problems, from the Gravitational Force to Two-body Scattering}.

\item Homeworks 5-9
\end{enumerate}

\noindent
\paragraph{Part 1 (50pt), the inverse-square law and the stability of planetary orbits.}
In homework 9 we studied an attractive potential
\[
V(r)=-\alpha/r,
\]

where the quantity $r$ is the absolute value of the relative position
and $\alpha$ is a constant.


When we rewrote the equations of motion in polar coordinates, we found the
analytical solution to the radial equation of motion 

\[
r(\phi) = \frac{c}{1+\epsilon\cos{(\phi)}},
\]
where $c=L^2/\mu\alpha$, with
the reduced mass $\mu$ and the angular momentum $L$, as
discussed during the lectures. With the transformation of a two-body
problem to the center-of-mass frame, the actual equations look like an
\emph{effective} one-body problem. 

The quantity $\epsilon$ is what we called the eccentricity. Since we
will mainly study bounded orbits, we have $0 \le \epsilon < 1$.  For
the Earth, the orbit is indeed close to circular and at perihelion
(the closest distance to the Sun), the Earth's center is about 0.98329
astronomical units (AU) or 147,098,070 km from the Sun's center. For
Earth, the orbital eccentricity is $\epsilon\approx 0.0167$. The outer
planets have more elliptical orbits. For example, Mars has its
perihelion at 206,655,215 km and its apehelion at 249,232,432 km.

In this part we will limit ourselves to the Earth-Sun system we
studied in homeworks 5 and 6. You can reuse your code with either the
Velocity-Verlet or the Euler-Cromer algorithms from homework 5 or 6.

This means also that $\alpha=GM_{\odot}M_{\mathrm{Earth}}$. We will
use $\alpha$ as a shorthand in the equations here. Keep in mind that
in homework 5 you scaled $GM_{\odot}=4\pi^2$ in your code.

The exercises here are all based on you analyzing the results from your code from homeworks 5, 6, 7 and 8.

As a reminder, we list the equations we studied in homeworks 5 and 6.
Newton's law of gravitation is given by a force $F_G$ (we assume this
is the force acting on Earth from the Sun)

\[
F_G=-\frac{GM_{\odot}M_{\mathrm{Earth}}}{r^2},
\]

where $M_{\odot}$ is the mass of the Sun and $M_{\mathrm{Earth}}$ is
the mass of the Earth. The gravitational constant is $G$ and $r$ is
the distance between the Earth and the Sun.  We assumed that the Sun
has a mass which is much larger than that of the Earth. We could
therefore safely neglect the motion of the Sun.

In homeworks 5 and 6 we assumed that the orbit of the Earth around the Sun 
was co-planar, and we took this to be the $xy$-plane.
Using Newton's second law of motion we got the following equations

\[
\frac{d^2x}{dt^2}=-\frac{F_{G,x}}{M_{\mathrm{Earth}}},
\]
and

\[
\frac{d^2y}{dt^2}=-\frac{F_{G,y}}{M_{\mathrm{Earth}}},
\]

where $F_{G,x}$ and $F_{G,y}$ are the $x$ and $y$ components of the
gravitational force.

You can obviously set $\alpha=GM_{\odot}M_{\mathrm{Earth}}$ as we did in homeworks 5 and 6.





\begin{itemize}
\item 1a (10pt) Use now your code from homework 5 (in cartesian coordinates). Start with a circular orbit setting $\epsilon=0$ and plot $x$ versus $y$. How would you choose the initial conditions to obtain a circular orbit?

\item 1b (10pt) Check that for the case of a circular orbit that both the kinetic and the potential energies are conserved. Why do we expect such a result if we have a circular orbit? 

\item 1c (10pt) With the same initial conditions (circular orbit) use Kepler's second law (see Taylor section 3.4) to show that angular momentum is conserved. Compare the value you get with the angular momentum you get from a circular orbit. 

\item 1d (10pt) Till now we have assumed that we have an inverse-square force $F(r) = -\alpha/r^2$. Let us rewrite this force as $F(r) =-\alpha/r^{\beta}$ with $\beta=[2,2.01,2.10,2.5,3.0,3.5]$. \textbf{Note}: in your code you are setting the force in say for example the $x$-direction (the same applies to the $y$ and eventual $z$-direction to $F(r) = -(\alpha/r^3)x$. It means that you study the dependence on the parameter $\beta$ you need to add $1$ to the power. Run your Sun-Earth code with these values of $\beta$ and plot $x$ versus $y$ (you can use the same initial conditions or switch to eliptical orbits). Discuss your
\end{itemize}

\noindent
results. Can you use the observations of planetary motion to determine by what amount Nature deviates from a perfect inverse-square law?



\begin{itemize}
\item 1e (10pt) Consider now an elliptical orbit with an initial position 1 AU from the Sun and an initial  velocity of 5 AU/yr. Show that the total energy is a constant (the kinetic and potential energies will vary). Show also that the angular momentum is a constant. If you change the parameter $\beta$ in $F(r) = -\alpha/r^{\beta}$ from $\beta=2$ to $\beta=3$, are these quantities conserved?  Discuss your results. (Hint: relate your results to Kepler's laws). 
\end{itemize}

\noindent
\paragraph{Part 2 (50pt), making a program for the solar system.}
Our final aim is to write a code which includes the known planets of the solar system. 

We will, as before, use so-called astronomical units when rewriting our equations. 
Using astronomical units (AU as abbreviation)it means that 
one astronomical unit of length, known as 1 AU, is the average distance between the Sun and Earth, that is
$1$ AU = $1.5\times 10^{11}$ m.  It can also be convenient to use years instead of seconds since years match
better the time evolution of the solar system. The mass of the Sun is $M_{\mathrm{sun}}=M_{\odot}=2\times 10^{30}$ kg. The masses of all relevant planets and their distances from the sun are listed in the table here in kg and AU.


\begin{quote}
\begin{tabular}{ccc}
\hline
\multicolumn{1}{c}{ Planet } & \multicolumn{1}{c}{ Mass in kg } & \multicolumn{1}{c}{ Distance to  sun in AU } \\
\hline
Earth   & $M_{\mathrm{Earth}}=6\times 10^{24}$ kg     & 1AU                    \\
Jupiter & $M_{\mathrm{Jupiter}}=1.9\times 10^{27}$ kg & 5.20 AU                \\
Mars    & $M_{\mathrm{Mars}}=6.6\times 10^{23}$ kg    & 1.52 AU                \\
Venus   & $M_{\mathrm{Venus}}=4.9\times 10^{24}$ kg   & 0.72 AU                \\
Saturn  & $M_{\mathrm{Saturn}}=5.5\times 10^{26}$ kg  & 9.54 AU                \\
Mercury & $M_{\mathrm{Mercury}}=3.3\times 10^{23}$ kg & 0.39 AU                \\
Uranus  & $M_{\mathrm{Uranus}}=8.8\times 10^{25}$ kg  & 19.19 AU               \\
Neptun  & $M_{\mathrm{Neptun}}=1.03\times 10^{26}$ kg & 30.06 AU               \\
Pluto   & $M_{\mathrm{Pluto}}=1.31\times 10^{22}$ kg  & 39.53 AU               \\
\hline
\end{tabular}
\end{quote}

\noindent
Pluto is no longer considered a planet, but we add it here for
historical reasons. It is optional in this midterm project to include
Pluto and eventual moons.

In setting up the equations we can limit ourselves to a co-planar
motion and use only the $x$ and $y$ coordinates. But you should feel
free to extend your equations to three dimensions, it is not very
difficult and the data from NASA are all in three dimensions.
You find these data at the 
\href{{http://www.nasa.gov/index.html}}{NASA} has an excellent site at \href{{http://ssd.jpl.nasa.gov/horizons.cgi#top}}{\nolinkurl{http://ssd.jpl.nasa.gov/horizons.cgi\#top}} site.

From there you can extract initial conditions in order to start your
differential equation solver.  At the above website you need to change
from \textbf{OBSERVER} to \textbf{VECTOR} and then write in the planet you are
interested in.  The generated data contain the $x$, $y$ and $z$ values
as well as their corresponding velocities. The velocities are in units
of AU per day.  Alternatively they can be obtained in terms of km and
km/s.

We will start with  the three-body problem, still with the Sun kept
fixed as the center of mass of the system but including Jupiter (the
most massive planet in the solar system, having a mass that is
approximately 1000 times smaller than that of the Sun) together with
the Earth. This leads to a three-body problem. Without Jupiter, the
Earth's motion is stable and unchanging with time. The aim here is to
find out first how much Jupiter alters the Earth's motion.


The program you have developed in homeworks 5 and 6 can easily be modified by
simply adding the magnitude of the force between the Earth and
Jupiter.

This force is given again by

\[
F_{\mathrm{Earth-Jupiter}}=-\frac{GM_{\mathrm{Jupiter}}M_{\mathrm{Earth}}}{r_{\mathrm{Earth-Jupiter}}^2},
\]

where $M_{\mathrm{Jupiter}}$ is the mass of Jupyter and
$M_{\mathrm{Earth}}$ is the mass of the Earth.  The gravitational constant
is $G$ and $r_{\mathrm{Earth-Jupiter}}$ is the distance between Earth
and Jupiter.

We assume again that the orbits of the two planets are co-planar, and
we take this to be the $xy$-plane (you can easily extend the equations
to three dimensions, feel free to run your calculations in two or three dimensions).

\begin{itemize}
\item 2a (20pt) Modify your coupled first-order differential equations from homework 5 in order to accomodate both the motion of the Earth and Jupiter by taking into account the distance in $x$ and $y$ between the Earth and Jupiter. Write out the differential equations for  Earth and Jupyter, keeping the Sun at rest (mass center of the system). Scale these equations in terms of Astronomical Units.

\item 2b (10pt) Use either the Euler-Cromer or Velocity-Verlet algorithms to compute the positions of the Earth and
\end{itemize}

\noindent
Jupiter. Repeat the calculations by increasing the mass of Jupiter by a factor of 10, 100 and 1000 and plot the position of the Earth.  Discuss your results and study the stability of this three-body system as function of the chosen
masses for Jupiter.

\begin{itemize}
\item 2c (50pt) Since the Sun is much more massive than all the other planets, we will define the Sun as our center of mass and set its velocity and position to zero.  Our final task is to add the remaining known planets and simulate the solar system asfunction of time. Add gradually one planet at the time. Develop a code which simulates the solar system with the above planets and plot their orbits. Discuss your results.
\end{itemize}

\noindent
\paragraph{Classical Mechanics Extra Credit Assignment: Scientific Writing and attending Talks.}
The following gives you an opportunity to earn \textbf{five extra credit
points} on each of the remaining homeworks and \textbf{ten extra credit points}
on the midterms and finals.  This assignment also covers an aspect of
the scientific process that is not taught in most undergraduate
programs: scientific writing.  Writing scientific reports is how
scientist communicate their results to the rest of the field.  Knowing
how to assemble a well written scientific report will greatly benefit
you in you upper level classes, in graduate school, and in the work
place.

The full information on extra credits is found at \href{{https://github.com/mhjensen/Physics321/blob/master/doc/Homeworks/ExtraCredits/}}{\nolinkurl{https://github.com/mhjensen/Physics321/blob/master/doc/Homeworks/ExtraCredits/}}. There you will also find examples on how to write a scientific article. 
Below you can also find a description on how to gain extra credits by attending scientific talks.


This assignment allows you to gain extra credit points by practicing
your scientific writing.  For each of the remaining homeworks you can
submit the specified section of a scientific report (written about the
numerical aspect of the homework) for five extra credit points on the
assignment.  For the two midterms and the final, submitting a full
scientific report covering the numerical analysis problem will be
worth ten extra points.  For credit the grader must be able to tell
that you put effort into the assignment (i.e.~well written, well
formatted, etc.).  If you are unfamiliar with writing scientific
reports, \href{{https://github.com/mhjensen/Physics321/blob/master/doc/Homeworks/ExtraCredits/IntroductionScientificWriting.md}}{see the information here}

The following table explains what aspect of a scientific report is due
with which homework.  You can submit the assignment in any format you
like, in the same document as your homework, or in a different one.
Remember to cite any external references you use and include a
reference list.  There are no length requirements, but make sure what
you turn in is complete and through.  If you have any questions,
please contact Julie Butler at butler@frib.msu.edu.


\begin{quote}
\begin{tabular}{ccc}
\hline
\multicolumn{1}{c}{ HW/Project } & \multicolumn{1}{c}{ Due Date } & \multicolumn{1}{c}{ Extra Credit Assignment } \\
\hline
HW 3               & 2-8           & Abstract                   \\
HW 4               & 2-15          & Introduction               \\
HW 5               & 2-22          & Methods                    \\
HW 6               & 3-1           & Results and Discussion     \\
\textbf{Midterm 1} & \textbf{3-12} & \emph{Full Written Report} \\
HW 7               & 3-22          & Abstract                   \\
HW 8               & 3-29          & Introduction               \\
HW 9               & 4-5           & Results and Discussion     \\
\textbf{Midterm 2} & \textbf{4-16} & \emph{Full Written Report} \\
HW 10              & 4-26          & Abstract                   \\
\textbf{Final}     & \textbf{4-30} & \emph{Full Written Report} \\
\hline
\end{tabular}
\end{quote}

\noindent

You can also gain extra credits if you attend scientific talks.
This is described here.


\paragraph{Integrating Classwork With Research.}
This opportunity will allow you to earn up to 5 extra credit points on a Homework per week. These points can push you above 100\% or help make up for missed exercises.
In order to earn all points you must:

\begin{enumerate}
\item Attend an MSU research talk (recommended research oriented Clubs is  provided below)

\item Summarize the talk using at least 150 words

\item Turn in the summary along with your Homework.
\end{enumerate}

\noindent
Approved talks:
Talks given by researchers through the following clubs:
\begin{itemize}
\item Research and Idea Sharing Enterprise (RAISE)​: Meets Wednesday Nights Society for Physics Students (SPS)​: Meets Monday Nights

\item Astronomy Club​: Meets Monday Nights

\item Facility For Rare Isotope Beam (FRIB) Seminars: ​Occur multiple times a week
\end{itemize}

\noindent
If you have any questions please consult Jeremy Rebenstock, rebensto@msu.edu.

All the material on extra credits is at \href{{https://github.com/mhjensen/Physics321/blob/master/doc/Homeworks/ExtraCredits/}}{\nolinkurl{https://github.com/mhjensen/Physics321/blob/master/doc/Homeworks/ExtraCredits/}}. 







% ------------------- end of main content ---------------

% #ifdef PREAMBLE
\end{document}
% #endif

