\documentstyle[a4wide,12pt]{article}

\begin{document}



\section*{Chaos in the driven nonlinear pendulum}

The angular equation of motion of the pendulum is given by
Newton's equation and with no external force it reads 
\begin{equation}
  ml\frac{d^2\theta}{dt^2}+mgsin(\theta)=0,
\end{equation}
with an angular velocity and acceleration given by
\begin{equation}
     v=l\frac{d\theta}{dt},
\end{equation}
and 
\begin{equation}
     a=l\frac{d^2\theta}{dt^2}.
\end{equation}


We do however expect that the motion will gradually come to an end
due a viscous drag torque acting on the pendulum. 
In the presence of the drag, the above equation becomes
\begin{equation}
   ml\frac{d^2\theta}{dt^2}+\nu\frac{d\theta}{dt}  +mgsin(\theta)=0,
\label{eq:pend1}
\end{equation}
where $\nu$ is now a positive constant parameterizing the viscosity
of the medium in question. In order to maintain the motion against
viscosity, it is necessary to add some external driving force. 
We choose here a periodic driving force. The last equation becomes then
\begin{equation}
   ml\frac{d^2\theta}{dt^2}+\nu\frac{d\theta}{dt}  +mgsin(\theta)=Asin(\omega t),
\label{eq:pend2}
\end{equation}
with $A$ and $\omega$ two constants representing the amplitude and 
the angular frequency respectively. The latter is called the driving frequency.



\begin{enumerate}
\item[a)] Rewrite Eqs.~(\ref{eq:pend1}) and (\ref{eq:pend2}) as dimensionless equations. That is, scale the equations.

\item [b)] Write then a code which solves  Eq.~(\ref{eq:pend1}) using the Euler-Cromer method and fourth-order
Runge Kutta method. Perform calculations for at least ten  periods with $N=100$, $N=1000$ and $N=10000$ mesh points
and values of $\nu = 1$, $\nu = 5$ and $\nu =10$.  
Set $l=1.0$ m, $g=1$ m/s$^2$ and $m=1$ kg.  
Choose as initial conditions $\theta(0) = 0.2$ (radians) and $v(0) = 0$ (radians/s). 
Make plots of $\theta$ (in radians) as function of time
and phase space plots of
$\theta$ versus the velocity $v$.
Check the stability of your results as functions of time and number of mesh points. 
Which case corresponds to damped, underdamped and overdamped oscillatory motion?  Comment your results.
\item[c)] Now we switch to Eq.~(\ref{eq:pend2}) for the rest of the project. Add an external driving force and 
set $l=g=1$, $m=1$, $\nu = 1/2$ and $\omega = 2/3$.  
Choose as initial conditions $\theta(0) = 0.2$ and $v(0) = 0$ and $A=0.5$ and $A=1.2$.
Make plots of $\theta$ (in radians) as function of time for at least 300 periods and phase space plots of
$\theta$ versus the velocity $v$. Choose an appropriate time step. Comment and explain the results for the different values of $A$.
\item[d)]  Keep now the constants from the previous exercise fixed but set now $A=1.35$, $A=1.44$ and 
$A=1.465$. Plot $\theta$ (in radians) as function of time for at least 300 periods for these values of $A$ and 
comment your results. 
\item[e)] We want to analyse further these results by making phase space plots of
$\theta$ versus the velocity $v$ using only the points where we have $\omega t=2n\pi$ where $n$ is an integer. These are normally
called the drive periods.
This is an example of what is called a Poincare section and is a very useful way to plot and analyze the behavior of a dynamical
system. Comment your results.
\end{enumerate}


\end{document}




