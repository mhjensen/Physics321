%%
%% Automatically generated file from DocOnce source
%% (https://github.com/hplgit/doconce/)
%%
%%
% #ifdef PTEX2TEX_EXPLANATION
%%
%% The file follows the ptex2tex extended LaTeX format, see
%% ptex2tex: https://code.google.com/p/ptex2tex/
%%
%% Run
%%      ptex2tex myfile
%% or
%%      doconce ptex2tex myfile
%%
%% to turn myfile.p.tex into an ordinary LaTeX file myfile.tex.
%% (The ptex2tex program: https://code.google.com/p/ptex2tex)
%% Many preprocess options can be added to ptex2tex or doconce ptex2tex
%%
%%      ptex2tex -DMINTED myfile
%%      doconce ptex2tex myfile envir=minted
%%
%% ptex2tex will typeset code environments according to a global or local
%% .ptex2tex.cfg configure file. doconce ptex2tex will typeset code
%% according to options on the command line (just type doconce ptex2tex to
%% see examples). If doconce ptex2tex has envir=minted, it enables the
%% minted style without needing -DMINTED.
% #endif

% #define PREAMBLE

% #ifdef PREAMBLE
%-------------------- begin preamble ----------------------

\documentclass[%
oneside,                 % oneside: electronic viewing, twoside: printing
final,                   % draft: marks overfull hboxes, figures with paths
10pt]{article}

\listfiles               %  print all files needed to compile this document

\usepackage{relsize,makeidx,color,setspace,amsmath,amsfonts,amssymb}
\usepackage[table]{xcolor}
\usepackage{bm,ltablex,microtype}

\usepackage[pdftex]{graphicx}

\usepackage[T1]{fontenc}
%\usepackage[latin1]{inputenc}
\usepackage{ucs}
\usepackage[utf8x]{inputenc}

\usepackage{lmodern}         % Latin Modern fonts derived from Computer Modern

% Hyperlinks in PDF:
\definecolor{linkcolor}{rgb}{0,0,0.4}
\usepackage{hyperref}
\hypersetup{
    breaklinks=true,
    colorlinks=true,
    linkcolor=linkcolor,
    urlcolor=linkcolor,
    citecolor=black,
    filecolor=black,
    %filecolor=blue,
    pdfmenubar=true,
    pdftoolbar=true,
    bookmarksdepth=3   % Uncomment (and tweak) for PDF bookmarks with more levels than the TOC
    }
%\hyperbaseurl{}   % hyperlinks are relative to this root

\setcounter{tocdepth}{2}  % levels in table of contents

% prevent orhpans and widows
\clubpenalty = 10000
\widowpenalty = 10000

% --- end of standard preamble for documents ---


% insert custom LaTeX commands...

\raggedbottom
\makeindex
\usepackage[totoc]{idxlayout}   % for index in the toc
\usepackage[nottoc]{tocbibind}  % for references/bibliography in the toc

%-------------------- end preamble ----------------------

\begin{document}

% matching end for #ifdef PREAMBLE
% #endif

\newcommand{\exercisesection}[1]{\subsection*{#1}}


% ------------------- main content ----------------------



% ----------------- title -------------------------

\thispagestyle{empty}

\begin{center}
{\LARGE\bf
\begin{spacing}{1.25}
PHY321 Classical Mechanics 1
\end{spacing}
}
\end{center}

% ----------------- author(s) -------------------------

\begin{center}
{\bf Final  project, due Friday April 30}
\end{center}

    \begin{center}
% List of all institutions:
\centerline{{\small midnight (1159pm)}}
\end{center}
    
% ----------------- end author(s) -------------------------

% --- begin date ---
\begin{center}
Apr 22, 2021
\end{center}
% --- end date ---

\vspace{1cm}


\subsection{Practicalities about  homeworks and projects}

\begin{enumerate}
\item You can work in groups (optimal groups are often 2-3 people) or by yourself. If you work as a group you can hand in one answer only if you wish. \textbf{Remember to write your name(s)}!

\item How do I(we)  hand in?  Due to the extraordinary situation we are in now, the final projec should be handed in fully via D2L. You can scan your handwritten notes and upload to D2L or you can hand in everyhting (if you are ok with typing mathematical formulae using say Latex) as a jupyter notebook at D2L. The numerical part should always be handed in as a jupyter notebook.
\end{enumerate}

\noindent
\paragraph{Introduction to the final project, total score: 120  points.}
The relevant reading background is
\begin{enumerate}
\item chapters 2-8 and 14 of Taylor

\item lecture notes throughout the semester and previous homework and midterm projects.
\end{enumerate}

\noindent
The final project aims at covering most of the topics we have
discussed during the semester. You should feel free to use either
paper and pencil and/or symbolic software (sympy, Mathematica or
similar) for the non-computational exercises.


\paragraph{Exercise 1, Two-body Problems and Conservative Forces (total 75pt).}
The relevant material from Taylor are chapters 4 and 8. Homework sets 6-9 and the chapters on \href{{https://mhjensen.github.io/Physics321/doc/LectureNotes/_build/html/chapter4.html}}{forces} and \href{{https://mhjensen.github.io/Physics321/doc/LectureNotes/_build/html/chapter6.html}}{two-body problems} from the lecture notes may also be of use. 

This exercise is a follow-up of hw6. There we studied the so-called
Lennard-Jones potential which is widely used in molecular dynamics
calculations. This potential is based on parametrizations from
expertiments. In molecular dynamics calculations the assumption is
that atoms move according to the laws of Newton, given the correct
model for interactions. We can say then that quantum-mechanical
degrees of freedom stemming from the interactions between electrons
and protons in an atom, are parametrized in terms of an effective
potential.

We will limit ourselves to a two-body problem.

The goal of this exercise is to model a gas of argon atoms (here two
atoms only interacting), where the atoms interact according to the
famous Lennard-Jones potential,

\begin{equation}
    V(r) = 4\varepsilon\left((\frac{\sigma}{r})^{12} - (\frac{\sigma}{r})^6\right), \label{eq:lj}
\end{equation}

where $r$ is the distance between two atoms,
$r=\vert\bm{r}_i-\bm{r}_j\vert$, that is the norm of the relative
distance vector $\bm{r}$. The quantities $\sigma$ and $\varepsilon$ are
parameters which determine which chemical compound is modelled. This
potential is a good approximation for noble gases like helium, neon, argon and other.

We start first with a basic study of the potential

\begin{itemize}
\item \textbf{1a (5pt):} Plot the potential as a function of $r$ with $\varepsilon=1$ and $\sigma=1$, for example for $r \in [0.9,3]$.

\item \textbf{1b (5pt):} The behaviour of $V(r)$ is vastly different for $r < \sigma$ and $r > \sigma$. Which term in the potential, (\ref{eq:lj}), dominates in each case and what is the effect?

\item \textbf{1c (5pt):} Find and characterize the equilibrium points of the potential.

\item \textbf{1d (5pt):} Describe qualitatively the motion of two atoms which start at rest separated by a distance of ${1.5}\sigma$. What if they start with a separation of ${0.95}\sigma$?  (Hint: use the graph of the potential.)
\end{itemize}

\noindent
Then we switch our attention to the equations of motion.

\begin{itemize}
\item \textbf{1e (5pt):} Find the force on atom $i$ at position $\bm{r}_i$ from atom $j$ at position $\bm{r}_j$. Is this a conservative force?

\item \textbf{1f (5pt):}  Are linear and angular momentum conserved? You need to show this by calculating the relevant quantities.

\item \textbf{1g (5pt):} Show that the equation of motion for atom $i$ is
\end{itemize}

\noindent
\[
  \frac{d^2\bm{r}_i}{dt^2} = \frac{24\varepsilon}{m} \sum_{j \neq i} \left(2(\frac{\sigma}{\vert\bm{r}_i-\bm{r}_j}\vert)^{12}-(\frac{\sigma}{\vert\bm{r}_i-\bm{r}_j}\vert)^6\right)\frac{\bm{r}_i-\bm{r}_j}{\vert\bm{r}_i-\bm{r}_j\vert^2}.
\]



Numerical accuracy is reduced when computing with values which are
many orders of magnitude apart. This is often an issue in physics, and
molecular dynamics is no exception. For example, the mass of argon is
smaller than 10E-25kg, while typical length scales are
on the order of nanometers, 10E-9m.

The remedy is to change units so that most quantities are close to
$1$. From (\ref{eq:lj}) it is clear that $\sigma$ and $\varepsilon$
are the typical scales for length and energy.

\begin{itemize}
\item \textbf{1h (5pt):} Introduce the scaled coordinates $\bm{r}_i\,'=\bm{r}_i/\sigma$ and show that the equation of motion can be rewritten in terms of these coordinates as (where $t'=t/\tau$ for a suitable choice of $\tau$.)
\end{itemize}

\noindent
\begin{equation}
\frac{d^2\bm{r}_i\,'}{{dt'^2}} = 24 \sum_{j \neq i} \left(2\vert\bm{r}_i\,'-\bm{r}_j\,'\vert^{-12}-\vert\bm{r}_i\,'-\bm{r}_j\,\vert^{-6}\right)\frac{\bm{r}_i\,'-\bm{r}_j\,'}{\vert\bm{r}_i\,'-\bm{r}_j\,'\vert^2}, \label{eq:undim}
\end{equation}
\begin{itemize}
\item \textbf{1i (5pt):} What is the characteristic time scale $\tau$, and what is its value for argon, which has $\sigma=3.405$Å (1Å=1E-10m), $m = 39.95u$ (with 1u=1.66E-27kg) and $\varepsilon=1.0318$ E-2eV (1eV=1.602E-19J)?
\end{itemize}

\noindent
We switch now to a numerical procedure and study the simulation of two interacting atoms.
\begin{itemize}
\item \textbf{1j (10pt):} Write a function which solves (\ref{eq:undim}) for two atoms and find the positions and velocities of the atoms as a function of time. Implement either the Euler-Cromer or the Velocity-Verlet method to solve the equations od motion.

\item \textbf{1k (5pt):} Simulate the motion of two atoms which start at rest separated by a distance of ${1.5}\sigma$. Use $\Delta t'={0.01}$, simulate until $t'=5$ and integrate with one of the above methods. Plot the distance between the atoms as a function of time. How does the motion fit with your expectations?

\item \textbf{1l (5pt):} Repeat the previous tasks, but now with an initial separation of $0.95\sigma$. Explain your results.

\item \textbf{1m (10pt):} Compute and plot the kinetic, potential and total energy as a function of time for. Should the total energy be conserved? Why, or why not? 
\end{itemize}

\noindent
\paragraph{Exercise 2, Coupled Harmonic Oscillators (45pt).}
The relevant chapters from Taylor are chapters 5-7 and the lectures notes on \href{{https://mhjensen.github.io/Physics321/doc/LectureNotes/_build/html/chapter5.html}}{harmonic oscillations} and the \href{{https://mhjensen.github.io/Physics321/doc/LectureNotes/_build/html/chapter8.html}}{Lagrangian Formalism and Calculus of Variations}. Your codes from homework sets 5-8 and the first midterm may also be of interest.

Consider a mass $m$ that is connected to a wall by a spring with
spring constant $k$. A second identical mass $m$ is connected to the
first mass by an identical spring. Motion is confined to the $x$ direction only.

\begin{itemize}
\item \textbf{2a (5pt):} Make a drawing of the system, set up forces and define variables $x_1$ and $x_2$ for the two masses.

\item \textbf{2b (5pt):} Write the Lagrangian in terms of the positions of the two masses $x_1$ and $x_2$.

\item \textbf{2c (5pt):} Use the Euler-Lagrange equations to find the analytical expressions for the  equations of motion.

\item \textbf{2d (10pt):} Find the analytical  solutions using a guess of  the type
\end{itemize}

\noindent
\[
x_1=Ae^{i\omega t},~~~x_2=Be^{i\omega t}.
\]
Solve for $A/B$ and $\omega$. Express your answers in terms of $\omega_0^2=k/m$.

\begin{itemize}
\item \textbf{2e (20pt):} Write now a program which solves these two coupled differential equations for $x_1$ and $x_2$. Compute the positions $x_1$ and $x_2$ by choosing your initial conditions and compare with the analytical answers from 2d.   
\end{itemize}

\noindent
\paragraph{Classical Mechanics Extra Credit Assignment: Scientific Writing and attending Talks.}
The following gives you an opportunity to earn \textbf{five extra credit
points} on each of the remaining homeworks and \textbf{ten extra credit points}
on the midterms and finals.  This assignment also covers an aspect of
the scientific process that is not taught in most undergraduate
programs: scientific writing.  Writing scientific reports is how
scientist communicate their results to the rest of the field.  Knowing
how to assemble a well written scientific report will greatly benefit
you in you upper level classes, in graduate school, and in the work
place.

The full information on extra credits is found at \href{{https://github.com/mhjensen/Physics321/blob/master/doc/Homeworks/ExtraCredits/}}{\nolinkurl{https://github.com/mhjensen/Physics321/blob/master/doc/Homeworks/ExtraCredits/}}. There you will also find examples on how to write a scientific article. 
Below you can also find a description on how to gain extra credits by attending scientific talks.


This assignment allows you to gain extra credit points by practicing
your scientific writing.  For each of the remaining homeworks you can
submit the specified section of a scientific report (written about the
numerical aspect of the homework) for five extra credit points on the
assignment.  For the two midterms and the final, submitting a full
scientific report covering the numerical analysis problem will be
worth ten extra points.  For credit the grader must be able to tell
that you put effort into the assignment (i.e.~well written, well
formatted, etc.).  If you are unfamiliar with writing scientific
reports, \href{{https://github.com/mhjensen/Physics321/blob/master/doc/Homeworks/ExtraCredits/IntroductionScientificWriting.md}}{see the information here}

The following table explains what aspect of a scientific report is due
with which homework.  You can submit the assignment in any format you
like, in the same document as your homework, or in a different one.
Remember to cite any external references you use and include a
reference list.  There are no length requirements, but make sure what
you turn in is complete and through.  If you have any questions,
please contact Julie Butler at butler@frib.msu.edu.


\begin{quote}
\begin{tabular}{ccc}
\hline
\multicolumn{1}{c}{ HW/Project } & \multicolumn{1}{c}{ Due Date } & \multicolumn{1}{c}{ Extra Credit Assignment } \\
\hline
HW 3               & 2-8           & Abstract                   \\
HW 4               & 2-15          & Introduction               \\
HW 5               & 2-22          & Methods                    \\
HW 6               & 3-1           & Results and Discussion     \\
\textbf{Midterm 1} & \textbf{3-12} & \emph{Full Written Report} \\
HW 7               & 3-22          & Abstract                   \\
HW 8               & 3-29          & Introduction               \\
HW 9               & 4-5           & Results and Discussion     \\
\textbf{Midterm 2} & \textbf{4-16} & \emph{Full Written Report} \\
HW 10              & 4-26          & Abstract                   \\
\textbf{Final}     & \textbf{4-30} & \emph{Full Written Report} \\
\hline
\end{tabular}
\end{quote}

\noindent

You can also gain extra credits if you attend scientific talks.
This is described here.


\paragraph{Integrating Classwork With Research.}
This opportunity will allow you to earn up to 5 extra credit points on a Homework per week. These points can push you above 100\% or help make up for missed exercises.
In order to earn all points you must:

\begin{enumerate}
\item Attend an MSU research talk (recommended research oriented Clubs is  provided below)

\item Summarize the talk using at least 150 words

\item Turn in the summary along with your Homework.
\end{enumerate}

\noindent
Approved talks:
Talks given by researchers through the following clubs:
\begin{itemize}
\item Research and Idea Sharing Enterprise (RAISE)​: Meets Wednesday Nights Society for Physics Students (SPS)​: Meets Monday Nights

\item Astronomy Club​: Meets Monday Nights

\item Facility For Rare Isotope Beam (FRIB) Seminars: ​Occur multiple times a week
\end{itemize}

\noindent
If you have any questions please consult Jeremy Rebenstock, rebensto@msu.edu.

All the material on extra credits is at \href{{https://github.com/mhjensen/Physics321/blob/master/doc/Homeworks/ExtraCredits/}}{\nolinkurl{https://github.com/mhjensen/Physics321/blob/master/doc/Homeworks/ExtraCredits/}}. 





% ------------------- end of main content ---------------

% #ifdef PREAMBLE
\end{document}
% #endif

